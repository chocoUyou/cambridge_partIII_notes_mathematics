\documentclass[10pt,a4paper]{report}


\usepackage{amsmath}
\usepackage[utf8]{inputenc}
\usepackage{amsmath}
\usepackage{amsfonts}
\usepackage{amssymb}
\usepackage{calrsfs}
\usepackage[left=2cm,right=2cm,top=2cm,bottom=2cm]{geometry}
\usepackage[mathscr]{euscript}
\usepackage{mathrsfs}

%%%%%%%%%for adding margins%%%%%%%%%
\usepackage{scrextend}
%%%%%%%%%%%%%%%%%%%%%%%%%%%%%%%

%for drawing commutative diagrams.
\usepackage{tikz-cd}  
%%%%%%%%%%%%%%%%%%%%%%%%%%%


%%%%%%%%%%%for wide hat%%%%%%%%%%%%
\usepackage{scalerel,stackengine}
\stackMath
\newcommand\reallywidehat[1]{%
\savestack{\tmpbox}{\stretchto{%
  \scaleto{%
    \scalerel*[\widthof{\ensuremath{#1}}]{\kern-.6pt\bigwedge\kern-.6pt}%
    {\rule[-\textheight/2]{1ex}{\textheight}}%WIDTH-LIMITED BIG WEDGE
  }{\textheight}% 
}{0.5ex}}%
\stackon[1pt]{#1}{\tmpbox}%
}

%\renewcommand{\hat}{\reallywidehat}
%%%%%%%%%%%%%%%%%%%%%%%%%%%%%%%%%%
\renewcommand{\hat}{\widehat}


%%%%%%%%%%for changing margin
\def\changemargin#1#2{\list{}{\rightmargin#2\leftmargin#1}\item[]}
\let\endchangemargin=\endlist 

\newenvironment{proof}
{\begin{changemargin}{1cm}{0.5cm} 
	}%your text here
	{\end{changemargin}
}
%%%%%%%%%%%%%%%%%%%%%%%%%%%%%


\begin{document}
\newcommand{\thm}{\textbf{Theorem) }}
\newcommand{\thmnum}[1]{\textbf{Theorem #1) }}
\newcommand{\defi}{\textbf{Definition) }}
\newcommand{\lem}{\textbf{Lemma) }}
\newcommand{\lemnum}[1]{\textbf{Lemma #1) }}
\newcommand{\prop}{\textbf{Proposition) }}
\newcommand{\propnum}[1]{\textbf{Proposition #1) }}
\newcommand{\cor}{\textbf{Corollary) }}
\newcommand{\cornum}[1]{\textbf{Corollary #1) }}


\newcommand{\pf}{\textbf{proof) }}
\newcommand{\eop}{\hfill  \textsl{(End of proof)} $\square$} %end of proof


\newcommand{\lap}{\triangle} %%Laplacian
\newcommand{\s}{\vspace{10pt}}
\newcommand{\bull}{$\bullet$}
\newcommand{\sta}{$\star$}
\newcommand{\reals}{\mathbb{R}}

\newcommand{\intN}{\mathbb{Z}_N}
\newcommand{\norms}[2]{\parallel #1 \parallel_{#2}}
\newcommand{\avg}{\mathbb{E}}
\newcommand{\prob}{\mathbb{P}}
\newcommand{\osc}{\text{\osc}}

\newcommand{\newday}{================================================================}

\renewcommand{\bar}{\overline}

\setlength\parindent{0pt}

\chapter*{Introduction to Discrete Analysis}
\s

\begin{proof}
\pf ddddddd
\end{proof}


=====================================================================================
(4th October 2018, Thursday)
\s

\section*{Chapter 1. The discrete Fourier transform}
\s

Let $N$ be some fixed positive integer. Write $\omega$ for $e^{2\pi i /N}$. and $\mathbb{Z}_N$ for $\mathbb{Z}/N\mathbb{Z}$. Let $f: \mathbb{Z}_N \rightarrow \mathbb{C}$. Given $r \in \mathbb{Z}_N$ define $\hat{f}(r)$ to be $\frac{1}{N} \sum_{x \in \mathbb{Z}_N} f(x) \omega^{-rx}$. From now on, we shall use the notation $\mathbb{E}_{x\in \mathbb{Z}_N}$ for $\frac{1}{N} \sum_{x \in \mathbb{Z}_N}$(so that $\hat{f}(r)  = \mathbb{E}_x f(x)e^{-2\pi i rx/N}$). If we write $\omega_r$ for the function $x\mapsto \omega^{rx}$ and $\langle f,g \rangle$ for $\mathbb{E}_x f(x)\bar{g(x)}$, then $\hat{f}(r) = \langle f,\omega_r \rangle$.
\s

Let us write $\norms{f}{p}$ for $(\mathbb{E}_x |f(x)|^p)^{1/p}$ and call the resulting space $L_p(\mathbb{Z}_N)$.
\s

\textbf{Important convention} : we use averages for the "original functions" in "physical space" and sums for their Fourier transforms in "frequency spaces".
\s

\lemnum{1} (Parseval's identity) If $f,g:\mathbb{Z}_N \rightarrow \mathbb{C}$, then $\langle \hat{f},\hat{g}\rangle =\langle f,g\rangle$.
\begin{itemize}
\item[ ] \pf
\begin{align*}
\langle \hat{f},\hat{g} \rangle &= \sum_{r} \hat{f}(r) \bar{\hat{g}(r)} = \sum_r (\mathbb{E}_x f(x)\omega^{-rx} \bar{(\mathbb{E}_y g(y) \omega^{-ry})} ) \\
&= \mathbb{E}_x \mathbb{E}_y f(x)\bar{g(y)} \sum_r \omega^{-r(x-y)} \\
&= \mathbb{E}_x \mathbb{E}_y f(x)\bar{g(y)} \Delta_{xy}\\
&=  \mathbb{E}_x f(x) \mathbb{E}_y \bar{g(y)} \Delta_{xy} = \mathbb{E}_x f(x)\bar{g(x)} = \langle f,g \rangle
\end{align*}

where $\Delta_{xy} = N$ if $x=y$ and $0$ if $x\neq y$.(in analogy with $\delta$-distribution).

\eop
\end{itemize}
\s

\lemnum{2}(Convolution identity) $\hat{f * g}(r) = \hat{f}(r) \hat{g}(r)$

\begin{itemize}
\item[ ]\pf The convolution $f*g(x)$ is defined to be $\avg_{y+z =x} f(y) g(z) = \avg_y f(y)g(x-y)$. Then,
\begin{align*}
\hat{f * g}(r) &= \avg_x f*g(x) \omega^{-rx} = \avg_x \avg_{y+z =x} f(y) g(z) \omega^{-rx} \\
&= \avg_x \avg_{y+z=x} f(y) g(z) \omega^{-ry}\omega^{-rz} = \avg_y f(y) \omega^{-ry} \avg_{z} g(z) \omega^{-rz} = \hat{f}(r) \hat{g}(r)
\end{align*}
\eop
\end{itemize} 
\s

\lemnum{3}(Inversion formula)  $f(x) = \sum_{r} \hat{f}(r)\omega^{rx}$.
\begin{itemize}
\item[ ] \pf
\begin{align*}
\sum_{r} \hat{f}(r) \omega^{rx} = \sum_r \avg_y f(y) \omega^{r(x-y)} = \avg_y f(y) \sum_r \omega^{r(x-y)} = \avg_y f(y) \Delta_{xy}  = f(x)
\end{align*}
\eop

\end{itemize}
\s

\textbf{Further Observations} :

\begin{itemize}
\item If $f$ is real-valued, then $\hat{f}(-r) = \avg_x f(x) \omega^{rx} = \bar{\avg_x f(x) \omega^{-rx}} = \overline{ \hat{f}(r) }$.

\item If $A\subset \intN$, write $A$(instead of $1_A$ or $\chi_A$) for the characteristic function of $A$. Then $\hat{A}(0) = \avg_x A(x) = \frac{|A|}{N}$, the \textbf{density} of $A$. Also, $\norms{\hat{A}}{2}^2 = \langle \hat{A}, \hat{A} \rangle> = \langle< A,A \rangle> = \avg_x A(x)^2 = |A|/N$.

\item Let $f: \intN \rightarrow \mathbb{C}$. Given $\mu \in \intN$ with $(\mu, N)=1$, define $f_{\mu}(x)$ to be $f(\mu^{-1}x)$. Then $\hat{f}_{\mu}(r) = \avg_x f_{\mu}(x)\omega^{-rx} = \avg_x f(x/\mu) \omega^{-rx} = \avg_x f(x)\omega^{-r\mu x} = \hat{f}(\mu r)$.
\end{itemize}

\subsection*{Roth's Theorem}
\s

\thmnum{4}(Szemer\'{e}di's theorem for $k=3$) For every $\delta >0$, there exists $N$ such that if $A \subset \{1, \cdots, N\}$ is a set of size at least $\delta N$, then $A$ must contain an arithmetic progression of length 3. 

\s

Basic strategy : show that if $A$ has density $\geq \delta$ and no 3AP, then there is a long arithmetic progression $P \subset \{1,\cdots,n\}$ such that
\begin{align*}
|A\cap P| \geq (\delta + c(\delta)) |P|
\end{align*}

====================================================================================
(9th October, Tuesday)
\s

\thmnum{4}(Roth's theorem, or Szemer\'{e}di's theorem for $k=3$) For every $\delta >0$, there exists $N$ such that if $A \subset \{1, \cdots, N\}$ is a set of size at least $\delta N$, then $A$ must contain an arithmetic progression of length 3. 
\s

\lemnum{5} Let $A,B,C \subset \intN$ have densities $\alpha,\beta,\gamma$. If $\max_{r\neq 0} |\hat{A}(r)| \leq \frac{\alpha(\beta \gamma)^{1/2}}{2}$ and $\frac{\alpha \beta \gamma}{2} > \frac{1}{N}$ then there exist $x, d \in \intN$ with $d\neq 0$ s.t. $(x,x+d,x+2d) \in A \times B \times C$.
\begin{proof}
\pf Assume $N$ is odd.

Consider the function $A(x)B(x+d)C(x+2d)$, which is non-zero if and only if there is a arithmetic progression of length 3. So $\avg_{x,d} A(x)B(x+d)C(x+2d)$ counts the density of $x,d$ satisfying such condition.
\begin{align*}
& \avg_{x,d}A(x)B(x+d)C(x+2d) = \avg_{x+z =2y}A(x)B(y)C(z) \\
=& \avg_{u}\big( \avg_{x+z =u} A(x)C(z) \big) \avg_{2y=u} B(y) = \avg_u A * C(u) B_2(u) = \langle A * C, B_2 \rangle \\
=& \langle \hat{A*C}, \hat{B}_2 \rangle = \langle \hat{A} \hat{C}, \hat{B}_2 \rangle = \sum_r \hat{A}(r) \hat{C}(r) \hat{B}(-2r) \\
=& \alpha \beta \gamma + \sum_{r\neq 0} \hat{A}(r) \hat{C}(r) \hat{B}(-2r)
\end{align*}
\textit{(Recall, $B_{\mu}(x)$ is defined to be $B(\mu^{-1}x)$ whenever $(\mu, N)=1$. We can generalize this to write $B_{\mu}(x) = \avg_{\mu u=x}B(u)$  )}

The first term $\alpha \beta \gamma$ gives a rough estimate of the value just in terms of the densities and the second term gives an estimate for how random the sets $A,B,C$ behave.
\begin{align*}
\big|\sum_{r\neq 0} \hat{A}(r) \hat{B}(-2r) \hat{C}(r) \big| &\leq \frac{\alpha (\beta \gamma)^{1/2}}{2} \sum_{r\neq 0} \big|\hat{B}(-2r)\big| \big|\hat{C}(r)\big| \\
&\leq \frac{\alpha (\beta \gamma)^{1/2}}{2} \big( \sum_r \big| \hat{B}(-2r) \big|^2 \big)^{1/2} \big( \sum_r \big| \hat{C}(r) \big|^2 \big)^{1/2} \quad \text{(Cauchy-Schwarz)} \\
& = \frac{\alpha (\beta \gamma)^{1/2}}{2} \norms{\hat{B}}{2} \norms{\hat{C}}{2} = \frac{\alpha (\beta \gamma)^{1/2}}{2} \norms{B}{2} \norms{C}{2} \quad \text{(Parseval)} \\
& = \frac{\alpha \beta \gamma}{2}
\end{align*}
The contribution to $\avg_{x,d}A(x)B(x+d)C(x+2d)$ from $d=0$ is at most $\frac{1}{N}$, so if $\frac{\alpha \beta \gamma}{2} > \frac{1}{N}$, we are done.

\eop
\end{proof}
\s

Now let $A$ be a subset of $\intN$ of density $\geq \delta$ and let $B=C = A\cap [\frac{N}{3},\frac{2N}{3}]$. 

\begin{itemize}
\item  If $B$ has density $< \frac{\delta}{5}$, then either $A \cap [1,\frac{N}{3}]$ or $A\cap [\frac{2N}{3},N]$ has density at least $\frac{2\delta}{5}$ so in that case we find an AP(arithmetic progression), $P$(one of $\intN \cap [1,N/3]$ and $\intN \cap [2N/3,N]$), of length about $N/3$ such that $|A\cap P| /|P| \geq 6\delta /5$.

\item Otherwise, if $\max_{r\neq 0} |\hat{A}(r)| \leq \frac{\delta^2}{10}$. We may pick $N$ large so that $\frac{\delta^3}{50} > \frac{1}{N}$. Then $A \times B \times C$ contains a 3AP by the \textbf{Lemma 5} and therefore $A$ contains a 3AP.

\item So if $A$ does not contain a 3AP, then either we find $P$ of length about $N/3$ with $|A \cap P|/|P| \geq 6\delta/5$ or we find $r\neq 0$ such that $|\hat{A}(r)| \geq \delta^2/10$. (This is sometimes called the \emph{dichotomy of order and chaos})
\end{itemize}

\s

\defi If $X$ is a finite set and $f: X \rightarrow \mathbb{C}$, $Y\subset X$, write $\text{osc}(f\big|_{Y})$ to mean $\max_{y_1,y_2 \in Y} |f(y_1) - f(y_2) |$.
\s

\lemnum{6} Let $r \in \hat{\mathbb{Z}}_N$(or just $\intN$) and let $\epsilon>0$. Then there is a partition of $\{ 1,2,\cdots, N\}$ into arithmetic progressions $P_i$ of length at least $c(\epsilon)\sqrt{N}$ such that $\text{osc}(\omega_r \big|_{P_i}) \leq \epsilon$ for each $i$.

\begin{proof}
\pf Let $t= \lfloor \sqrt{N} \rfloor$. Of the numbers $1,\omega^r,\cdots,\omega^{tr}$ there must be two that differ by at most $2\pi/t$(by sort-of-pigeon hole principle.) If $|\omega^{ar} - \omega^{br} | \leq 2\pi /t$ with $a<b$, then $|1-\omega^{dr}| \leq 2\pi /t$ where $d= b-a$. Now, by the triangular inequality, if $u<v$ then
\begin{align*}
|\omega^{urd} - \omega^{vrd} | \leq |\omega^{urd} - \omega^{(u+1)rd} | + \cdots + |\omega^{(v-1)rd} - \omega^{vrd} | \leq \frac{2\pi}{t} (v-u)
\end{align*}
So if $P$ is a progression with common difference $d$ and length $l$, then $\text{osc} (\omega_{r} \big|_{P}) \leq \frac{2\pi l}{t}$. So divide up $\{1,\cdots, N\}$ into residue classes root $d$ and partition each residue class into parts of length between $\frac{\epsilon t}{4\pi}$ and $\frac{\epsilon t}{2\pi}$(this is possible, since $d\leq t \leq \sqrt{N}$.) We are done, with $c(\epsilon) = \frac{\epsilon}{16}(<\frac{\epsilon}{4\pi})$.

\eop
\end{proof}
\s

Now let us use the  information that $r\neq 0$ and $|\hat{A}(r) | \geq \delta^2/10$.

Define the \textbf{balanced function} $f$ of $A$ by $f(x) = A(x) - |A|/N$ for each $x$. Note that $\hat{f}(0) =0 $ and $\hat{f}(r) = \hat{A}(r)$ for all $r\neq 0$.

Now let $P_1, \cdots, P_m$ be given by \textbf{Lemma 6} with $\epsilon = \delta^2/20$. Then
\begin{align*}
\frac{\delta^2}{10} &\leq |\hat{f}(r)| = \frac{1}{N} \big| \sum_x f(x) \omega^{-rx} \big| \leq \frac{1}{N} \sum_i \big| \sum_{x\in P_i}  f(x) \omega^{-rx} \big | \\
&\leq \frac{1}{N} \sum_{i=1}^m \Big[ \big| \sum_{x\in P_i} f(x) \omega^{-rx_i} \big| + \big| \sum_{x\in P_i} f(x)(\omega^{-rx} - \omega^{-rx_i})   \big|     \Big] \quad \text{where } x_i \text{ arbitrary} \in P_i \\
&\leq \frac{1}{N} \sum_{i=1}^N \big| \sum_{x\in P_i} f(x) \big| + \frac{\delta^2}{20}
\end{align*}
So $\sum_{i=1}^n \big| \sum_{x_i\in P_i} f(x) \big| \geq \delta^2 N/20$. Also, $\sum_{i=1}^m \sum_{x\in P_i} f(x) =0$. So
\begin{align*}
\sum_{i=1}^m \Big(  \big| \sum_{x\in P_i} f(x) \big| + \sum_{x\in P_i}f(x) \Big) \geq \frac{\delta^2}{20} \sum_{i=1}^m |P_i|
\end{align*}
Therefore, $\exists i$ s.t. $|\sum_{x\in P_i} f(x) | + \sum_{x\in P_i}f(x) \geq \frac{\delta^2}{20} |P_i|$ by pigeon hole principle and hence $\sum_{x\in P_i}f(x) \geq \frac{\delta^2}{40}|P_i|$. This implies $|A\cap P_i | = \big| \sum_{x\in P_i} f(x) + \frac{|A|}{N}  \big|   \geq (\delta + \delta^2/40)|P_i|$
\s

\newday

(11th October, 2018)
\s

(Example sheets handed out - do not try to use particular theorem, but to apply the idea of using Fourier analysis in combinatorics.

Why does Fourier analysis have anything to do with combinatorics? Because we can formulate various problems in terms of Fourier transformation, e.g. by convolution and Parseval's law.)
\s

\bull What we have showed : Let $A \subset \intN$, $|A|\geq \delta N$. Then,
\begin{itemize}
\item[(1)] $A$ contains a 3AP \emph{or}
\item[(2)] $N$ is even \emph{or}
\item[(3)] $\exists P \subset \{ 1,\cdots, N\}$, $|P|\geq N/3$ s.t. $|A\cap P| \geq \frac{6\delta}{5} |P|$ \emph{or}
\item[(4)] $\exists P \subset \{1,\cdots, N\}$, $|P| \geq \frac{\delta^2}{320} \sqrt{N}$ s.t $|A\cap P| \geq (\delta + \delta^2/40)|P|$
\end{itemize}

\quad If (2) holds, write $N=N_1+N_2$ with $N_1, N_2$ odd, $N_1, N_2 \simeq N/2$. Then $A$ has density at least $\delta$ in one of $\{1,\cdots, N_1\}$ or $\{N_1 +1, \cdots, N_1 + N_2\}$.

\quad If (4) holds (NB (3)$\Rightarrow$ (4)) then we pass to $P$ and start again(that is, replace $\intN$ with $P$ and $A$ with $A\cap P$). After $40/\delta$ iterations, the density at least doubles. So the total number of iterations we can have is at most $\leq \frac{40}{\delta} + \frac{40}{2\delta} + \frac{40}{4\delta} + \cdots \leq \frac{80}{\delta}$.

\quad If $\frac{\delta^2}{320} \sqrt{N} \geq N^{1/3}$ at each iteration, and $\delta^3/25 \geq N^{-1}$ (which follow from the first condition) then after $\frac{80}{\delta}$ iterations, we have is $N\geq N^{(1/3)^{80/\delta}}$. So the argument works provided
\begin{align*}
&N^{(1/3)^{80/\delta}} \geq \Big( \frac{320}{\delta^2} \Big)^6 \Leftrightarrow \Big( \frac{1}{3} \Big)^{80/\delta} \log N \geq 6(\log 320 + 2\log \frac{1}{\delta}) \\
\Leftrightarrow \quad & \frac{-80}{\delta} \log 3 + \log \log N \geq \log 6 + \log (\log 320 + 2\log \frac{1}{\delta}) \\
\Leftarrow \quad &\log \log N \geq \frac{160}{\delta} \\
\Leftarrow \quad &\delta\geq \frac{160}{\log \log N}
\end{align*}
(just the follow the idea of calculation,,, do not try to learn a particular thing from this calculation)

(In fact 320 above should be corrected to 640 because of (2))
\s

\subsection*{Bogolyubov's Method}

\defi Let $K \subset \hat{\mathbb{Z}}_N$ and let $\delta >0$. The \textbf{Bohr set} $B(K,\delta)$ has two definitions.
\begin{itemize}
\item[(1)] $B(K,\delta) = \{x\in \intN : rx \in [-\delta N, \delta N] \quad \forall r\in K \}$ (arc-length definition)
\item[(2)] $B(K,\delta) = \{x\in \intN : |1- \omega^{rx}| \leq \delta \quad \forall r\in K|\}$ (chord-length definition)
\end{itemize}
\s

It turns out that Bohr set has a lot of structure.
\s

\defi Let $G$ be an Abelian group and let $A,B$ be subsets of $G$. Then 
\begin{align*}
A+B = \{ a+b: a\in A, b\in B\} \\
A-B = \{a-b: a\in A, b\in B\} \\
rA = \{a_1+ \cdots + a_r : a_1,\cdots,a_r \in A  \}
\end{align*}
\s

\lemnum{7} Let $A\subset \intN$ be a set of density $\alpha$. Then $2A-2A$ contains a Bohr set $B(K,1/4)$(arc) with $|K|\leq \alpha^{-2}$.

\begin{proof}
\pf Observe that $x\in 2A-2A$ if and only if $A*A*(-A)*(-A)(x) \neq 0$(think for a second). But
\begin{align*}
A*A*(-A)*(-A)(x) &= \sum_{r} \reallywidehat{A*A*(-A)*(-A)}(r) \omega^{rx} \quad \text{(inversion formula)} \\
&= \sum_r  \big| \hat{A}(r) \big|^4 \omega^{rx}
\end{align*}
Let $K = \{r: \big| \hat{A}(r) \big| \geq \alpha^{3/2} \}$. Then 
\begin{align*}
\alpha = \norms{\hat{A}}{2}^2 = \sum_{r} \big| \hat{A}^2(r) \big|^2 \geq \alpha^3 |K|
\end{align*}
so $|K| \leq \alpha^{-2}$.
\quad Now suppose that $x\in B(K,1/4)$. Then 

\begin{align*}
\sum_{r} \big| \hat{A}(r)\big|^4 \omega^{rx}  = \alpha^4 + \sum_{r\in K,r\neq 0} \big| \hat{A}(r)\big|^4 \omega^{rx} + \sum_{r \notin K} \big| \hat{A}(r)\big|^4 \omega^{rx}
\end{align*}
The real part of the second term is non-negative, since $rx\in [-N/4,N/4]$ when $r\in K$. Also the final term can be bounded by
\begin{align*}
\Big| \sum_{r \notin K} \big| \hat{A}(r)\big|^4 \omega^{rx} \Big| \leq \sum_{r \notin K} \big| \hat{A}(r)\big|^4 < \alpha^3 \sum_{r \notin K} \big| \hat{A}(r)\big|^2 \leq \alpha^4
\end{align*}
It follows that $\text{Re}\Big( \sum_r  \big| \hat{A}(r) \big|^4 \omega^{rx} \Big) >0$, so $x\in 2A-2A$.

\eop
\end{proof}
\s

\lemnum{8} Let $K \subset \intN$ and let $\delta >0$. Then
\begin{itemize}
\item[(i)] $B(K,\delta)$ has density at least $\delta^{|K|}$.
\item[(ii)] $B(K,\delta)$ contains a mod-$N$ arithmetic progression of length $\geq \delta N^{1/|K|}$.
\end{itemize}
\begin{proof}
\pf

\begin{itemize}
\item[(i)]Let $K = \{r_1,\cdots, r_k\}$. Consider the $N$ number of $k$-tuples $(r_1 x, \cdots, r_k x)\in \intN^k$(for $x\in \intN$). If we intersect this set of $k-tuples$ with a random "box" $Q=[t_1,t_1+\delta N] \times \cdots \times [t_k,t_k+\delta N]$. Then the expected number of $k$-tuples in the box is $\delta^k N$(since each intersection has probability $\delta^k$.)

\quad But if $(r_1 x,\cdots, r_k x)$  and $(r_1 y,\cdots, r_k y)$ belong to this box, then $x-y \in B(K,\delta)$, so the expected number of in $(r_1 y,\cdots, r_k y)$ is smaller or equal to the expected number of $y$ in $B(K,\delta)$. i.e. if we let $q$ be a random point in $Q \cap \intN^k$, then
\begin{align*}
\avg( \sum_y 1_{y\in B(K,\delta)}) \geq \avg(\sum_{y} 1_{q \in Q} 1_{q + y  \in Q}) = \sum_y \prob ( y\in Q  ) = N |\delta|^{K}
\end{align*}
(there must be an easy argument... in fact, in the lecture, it was concluded before ~so the expected number of, so there must be more intuitive way of understanding this)
\item[(ii)] If we take $\eta > N^{-1/k}$, then by (i) we get that $|B(K,\eta)|>1$ so $\exists x \in B(K,\eta)$ s.t. $x\neq 0$. But then $dx \in B(K,d\eta)$ for every $d$. So if $d\eta \leq \delta$ then $B(K,d\eta) \subset B(K,\delta)$ so $dx\in B(K,\delta)$, that gives us an AP of length at least $\delta/\eta$. So we get one of length at least $\delta N^{1/k}$.
\end{itemize} 
\eop
\end{proof}
\s

\newday

(16th October, Tuesday)
\s

(about the example sheet - Question 2, has to fix $\alpha^2 \beta^2 N^4$ to $\alpha^2 \beta^2 N^3$ : a natural way to think about this is "normalizing". We decide one element from other three, so the factor of $N$ should be 3.)
\s

Recall : $B(K,\delta) = \{x\in \intN : rx\in [-\delta N, \delta N]$ for all $r\in K \}$
\s

\defi Let $A$, $B$ be subsets of Abelian groups and let $\phi : A\rightarrow B$. Then $\phi$ is a \textbf{Freiman homomorphism of order $k$} if
\begin{align*}
a_1 + \cdots + a_k = a_{k+1} + \cdots + a_{2k} \quad \Rightarrow \phi(a_1) + \cdots \phi(a_k) = \phi(a_{k+1}) +\cdots +\phi(a_{2k})
\end{align*}
If $k=2$, we call this a \textbf{Freiman homomorphism}. In that case, the condition is equivalent to
\begin{align*}
a-b = c-d \quad \Rightarrow \phi(a) - \phi(b) = \phi(c) - \phi(d)
\end{align*}

\quad If $\phi$ has an inverse which is also a Freiman homomorphism of order $k$ then $\phi$ is a Freiman isomorphism or order $k$>


\s

Consider $\begin{matrix}
\cdot & \cdot & \cdot \\
\cdot & \cdot & \cdot \\
\cdot & \cdot & \cdot
\end{matrix}
$ and $\cdots$, $\cdots$, $\cdots$. They have same additive structure, in some sense, and the Freiman isomorphism describes this.
\s

\renewcommand{\vec}{\underline}

\lemnum{9} Assume $0 \notin K$ and $N$ prime. If $\delta <1/4$, then $B(K,\delta)$ (arc) is Freiman isomorphic to the intersection in $\reals^{|K|}$ of $[-\delta N, \delta N]^|K|$ with some lattice $\Lambda$. 
\begin{proof}
\pf Let $K = \{r_1, \cdots, r_k\}$ and let $\Lambda = N \mathbb{Z}^k + \{(r_1x, \cdots, r_kx) : x\in \mathbb{Z}\}$. Write $\vec{r}$ for $(r_1, \cdots, r_k)$. \emph{Claim} that $B(K,\delta) \cong\Lambda \cap [-\delta N, \delta N]^k$.
\s

\quad Define a map $\phi : B(K,\delta) \rightarrow \Lambda \cap [-\delta N,\delta N]^k$ by sending $x$ to $(\langle r_1 x\rangle, \cdots, \langle r_k x\rangle)$ where $\langle u\rangle$ means the least-modulus residue of $u$ mod $N$. If $x+ y = z+w$, then $\vec{r} x+ \vec{x} y = \vec{r} z + \vec{r} w$ in $\intN^k$. But for each $i$, $\langle r_i x \rangle + \langle r_i y \rangle - \langle r_z \rangle- \langle r_i w \rangle \in [-4\delta N, 4\delta N]$. Since $\delta < 1/4$, that implies that $\langle r_i x \rangle + \langle r_i y \rangle - \langle r_z \rangle- \langle r_i w \rangle =0$. So $\langle \vec{r} x \rangle + \langle \vec{r} y \rangle - \langle \vec{r} \rangle- \langle \vec{r} w \rangle$.

\quad That already implies that $\phi$ is an injection.

\quad If $\vec{r} x + \vec{a} N \in [-\delta N, \delta N]^k$(this is a way of writing a typical lattice), then $r_i x \in [-\delta N, \delta N]$ mod $N$ for each $i$, so $x\in B(K,\delta)$ and $\phi(x) = \vec{r}x + \vec{a}N$. So $\phi$ is a surjection.

\quad If $\vec{r} x+ \vec{a} N + \vec{r} y + \vec{b} N = \vec{r} z + \vec{c}N + \vec{ r} w + \vec{d}N$, then $r_i(x+y) = r_i (z+w)$ mod $N$, s $x+y = z+w$ mod $N$, so the inverse of $\phi$ is also a Freiman homomorphism.

\eop 
\end{proof}
\s

(we will not prove in this course, but later in next term, a generalized version of AP contained in an intersection of $K$ and a box - this requires more knowledge in geometry of numbers, so cannot be dealt here)
\s

\lemnum{10} Let $\Lambda$ be a lattice and let $C$ be a symmetric convex body, both in $\reals^k$. Then $|\Lambda \cap C | \leq 5^k |\Lambda \cap \frac{C}{2}|$

(5 is not an optimal bound, but convenient to prove - how do we reduce this?)
\begin{proof}
\pf Let $x_1, \cdots, x_n$ be a maximal subset of $\Lambda \cap C$ such that for all $i \neq j$, $x_j \notin x_i + \frac{C}{2}$. Then by maximality, the sets $x_i + C/2$ cover all of $\Lambda \cap C$. Also, the sets $x_i + C/4$ are disjoint subsets of $\reals^k$, and they are all contained in $C + \frac{C}{4} = 5C/4$. So $m \leq \frac{\text{vol}(5C/5)}{\text{vol}{C/4}} = 5^k$. 
\end{proof}
\s

\cornum{11} If $N$ is primes, $0\notin K$, $|K| =k$, $\delta <1/4$, then $|B(K,\delta)| \delta 5^k |B(k,\delta/2)|$
\s

\section*{2. Sum Sets and their Structure}

Suppose $A$ is a subset of integers, and has $k$ such that $|A+A | \leq k|A|$. What can we say about the set $A$? What can we say about the bound of $|rA-sA|$? (will prove $|rA-sA| \leq k^{r+s})|A|$)
\s

\lemnum{1} (George Petridis - a former student of T.W.Gowers) Let $A_0$ and $B$ be finite subsets of an Abelian group such that $|A_0 + B | \leq K_0 |A_0| $. Then there exists a subset $A\subset A_0$ and $K\leq K_0$ such that $|A+B+C|\leq K|A+C|$ for every finite subset $C$ of the group.
\s

-the style of proof is so different from other combinatorics proofs. It uses induction on $C$.(Lesson to take : do not dispose a method for proof even before thinking about it)
\begin{proof}
\pf Let $A$ minimize the ration $|A+B|/|A|$ and elt the minimal ration be $K$.

\bull Claim : this works. We prove this by induction on $C$.

\quad If $C=\phi$, we are all happy. Now assume it for $C$ and let $x\notin C$.

Then $A+(C \cup \{x\}) = (A+C) \cup \Big[ (A+x) \backslash (A'+x) \Big]$ where $A' = \{a\in A: a+x\in A+ C\}$. This is a disjoint union, so
\begin{align*}
|A+(C\cup \{x\})| = |A+C| + |A| - |A'|
\end{align*}
Also, $A+B + (C\cup \{x\}) = (A+B+C) \cup \big( (A+B+x) \backslash (A'+B+x)\big)$ since if $a+x\in A+C$, then $a+B+x \subset A+B+C$.
\begin{align*}
|A+B+(C\cup \{x \}) | &\leq |A+B+C| + |A+B| - |A'+B| \\
&\leq K|A+C| + K|A| - K|A'|
\end{align*}
by induction and minimality property of $A$

\eop
\end{proof}






















\end{document}