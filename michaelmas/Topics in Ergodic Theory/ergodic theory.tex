\documentclass[10pt,a4paper]{report}


\usepackage{amsmath}
\usepackage[utf8]{inputenc}
\usepackage{amsmath}
\usepackage{amsfonts}
\usepackage{amssymb}
\usepackage{calrsfs}
\usepackage[left=2cm,right=2cm,top=2cm,bottom=2cm]{geometry}
\usepackage[mathscr]{euscript}

%%%for drawing commutative diagrams.%%%%%%
\usepackage{tikz-cd}  
%%%%%%%%%%%%%%%%%%%%%%%%%%%%%%%%%%%%%%%%%%

%%%%%%%%%%for changing margin
\def\changemargin#1#2{\list{}{\rightmargin#2\leftmargin#1}\item[]}
\let\endchangemargin=\endlist 

\newenvironment{proof}
{\begin{changemargin}{1cm}{0.5cm} 
	}%your text here
	{\end{changemargin}
}

\newenvironment{subproof}
{\begin{changemargin}{0.5cm}{0.5cm}
	}%your text here
	{\end{changemargin}
}
%%%%%%%%%%%%%%%%%%%%%%%%%%%%%

\begin{document}
\newcommand{\thm}{\textbf{Theorem) }}
\newcommand{\thmnum}[1]{\textbf{Theorem #1) }}
\newcommand{\defi}{\textbf{Definition) }}
\newcommand{\lem}{\textbf{Lemma) }}
\newcommand{\lemnum}[1]{\textbf{Lemma #1) }}
\newcommand{\prop}{\textbf{Proposition) }}
\newcommand{\pf}{\textbf{proof) }}
\newcommand{\cor}{\textbf{Corollary) }}
\newcommand{\cornum}[1]{\textbf{Corollary #1) }}

\newcommand{\lap}{\triangle} %%Laplacian
\newcommand{\s}{\vspace{10pt}}
\newcommand{\bull}{$\bullet$}
\newcommand{\sta}{$\star$}
\newcommand{\reals}{\mathbb{R}}

\newcommand{\eop}{\hfill  \textsl{(End of proof)} $\square$} %end of proof

\newcommand{\intN}{\mathbb{Z}_N}
\newcommand{\norms}[2]{\parallel #1 \parallel_{#2}}
\newcommand{\abs}[1]{\big| #1 \big|}
\newcommand{\avg}{\mathbb{E}}
\newcommand{\borel}{\mathscr{B}}
\newcommand{\setlimsup}[2]{\bigcap_{#1=1}^{\infty}\bigcup_{#2=#1}^{\infty}}
\newcommand{\dlim}{D-\lim}
\newcommand{\clim}{C-\lim}

\newcommand{\newday}{======================================================================}

\setlength\parindent{0pt}
\noindent

\chapter*{Topics in Ergodic Theory}
\s

=====================================================================================
(4th October 2018, Thursday)
\s

\section*{1. Measure preserving system}

\textbf{Measure preserving system} : $(X, \borel, \mu, T)$, where $X$ is a set, $\borel$ is a $\sigma$-algebra, $\mu$ is a probability measure with $\mu(A) \geq 0$ $\forall A \in \borel$, $\mu(X)=1$, and $T$ is a measure-preserving transformation. That is, $T: X\rightarrow X$ is measurable s.t. $\mu(T^{-1}(A)) = \mu(A)$ $\forall A \in \borel$.
\s

If $Y$ is a random element of $X$ with distribution $\mu$, then $T(Y)$ also has distribution $\mu$.
\s

\textbf{Example)}

\begin{itemize}
\item (Circle rotation) Let $X = \reals/ \mathbb{Z}$, $\borel$ be the Borel sets, $\mu$ be the Lebesgue measure and $T = R_{\alpha}$ where $R_{\alpha}(x) = x+\alpha$, and $\alpha \in \reals/\mathbb{Z}$ is parameter.

\item (Times 2 map) $X = \reals/\mathbb{Z}$, $\borel$ be the Borel sets, $\mu$ is a Lebesgue measure, $T = T_2$ where $T_2(x) = 2x$.

\begin{itemize}
\item[ ] \textsl{(proof that $T_2$ is measure preserving)} First prove for intervals : let $I = (a,b)$. Then $\mu(I) = b-a$ and $\mu(T_2^{-1} I)= \mu((\frac{a}{2},\frac{b}{2}) \cup (\frac{a+1}{2},\frac{b+1}{2})) = b/2 - a/2 + b/2 - a/2 = b-a$.(Just use Dynkin's lemma to conclude... Or,)

Now let $U \subset \reals / \mathbb{Z}$ be open. Then $U = \sqcup_{i} I_i$ is a disjoint union of intervals, so
\begin{align*}
\mu(T^{-1}U) = \mu(\sqcup_j T^{-1}I_j) = \sum_j \mu(T^{-1} I_j) = \sum_{j} \mu(I_j) = \mu(U)
\end{align*}

Let $K \subset \reals/\mathbb{Z}$ be a compact set. Then
\begin{equation*}
\mu(T^{-1}K) = 1- \mu((T^{-1}K)^c) = 1- \mu(T^{-1}(K^c)) = 1- \mu(K^c) = \mu(K)
\end{equation*}

Let $A$ be an arbitrary Borel set and let $\epsilon >0$. Then $\exists U$ open and $\exists K$ compact such that $K\subset A \subset A$ and $\mu(U \backslash K) < \epsilon$, so
\begin{align*}
\mu(K) = \mu(T^{-1} K) \leq \mu(T^{-1} A) \leq \mu(T^{-1} U) = \mu(U)
\end{align*}

We also have $\mu(K) \leq \mu(A) \mu(U)$. Since $\mu(U) - \mu(K) < \epsilon$, $|\mu(A) - \mu(T^{-1}A) | < \epsilon$. Since $\epsilon$ was arbitrary, so $\mu(A) = \mu(T^{-1} A)$.

\eop
\end{itemize}
\end{itemize}
\s

The \textbf{orbit} $x \in X$ is the sequence $x, Tx, T^2 x, \cdots$.
\s

Some Questions:

\begin{itemize}
\item Let $A \in \borel$ and $ x\in A$. Doest the orbit of $x$ visit $A$ infinitely often?

\item What is the proportion of the times $n$ such that $T^n x$ is in $A$? 

\item (Mixing property) What is $\mu(\{x \in A : T^nx \in A\})$ if $n$ is large
\end{itemize}
\s

\textbf{Example)} Let $A = [0,\frac{1}{4}) \subset \reals/\mathbb{Z}$ and $T = T_2$. Then $T^n x \in A \Leftrightarrow (n+1)^{\text{st}}$ and $(n+2)^{\text{nd}}$ binary digits of $x$ are 0.

For example, $x = 1/6 = 0.00101010\ldots_{(2)}$ never comes back to $A$.

Another interesting fact : $\mu(\{x : x\in A, T_2^n x \in A \}) = 1/16$ if $n\geq2$.(Circle rotation has very different property.)
\s

\subsection*{Markov Shift}

\begin{itemize}
\item Let $(p_1, p_2, \cdots, p_n)^T$ be a probability vector. Let $A \in \reals_{\geq 0}^{n\times n}$ be the \textbf{matrix of transition probabilities}.

Assumptions : (1) $A(1,\cdots,1)^T = (1,\cdots,1)^T$; (2) $(p_1, \cdots, p_n) A = (p_1, \cdots, p_n)$

\item Let $X = \{ 1,\cdots,n\}^{\mathbb{Z}}$, $\borel$ be the Borel $\sigma$-algebra generated by the product topology of the discrete topology on $\{1,\cdots, n\}$, and $T = \sigma$ is the shift map $(\sigma X)_m = X_{m+1}$.

\item Let $\mu(\{x\in X : x_m=i_0,\cdots,x_{m+n}=i_n\}) = p_{i_0} a_{i_0 i_1} \cdots a_{i_{n-1} i_n}$.
\end{itemize}
\s

\section*{2. Furstenberg's correspondence principle}

\thm (Szemer\'{e}di) Let $S \subset \mathbb{Z}$ of positive upper Banach density. That is:
\begin{align*}
\bar{d}(S) = \limsup_{N,M : M-N \rightarrow \infty} \frac{1}{M-N}|S \cap [N,M-1] | >0.
\end{align*}

Then $S$ contains arbitrary long arithmetic progressions. That is, $\forall l$, $\exists a \in \mathbb{Z}$, $d\in \mathbb{Z}_{>0}$ such that $a, a+d, \cdots, a+(l-1)d \in S$.
\s

\thm (Furstenberg) (Multiple recurrence) Let $(X,\borel,\mu, T)$ be a MPS(Measure preserving system). Let $A \in \borel$ be s.t. $\mu(A) >0$. Let $l\in \mathbb{Z}_{>0}$. Then
\begin{align*}
\liminf_{N\rightarrow \infty} \frac{1}{N} \sum_{n=1}^N \mu(A\cap T^{-n}(A) \cap \cdots \cap T^{-(l-1)n}(A))>0
\end{align*}
\s

=====================================================================================
(6th October 2018, Saturday)
\s

\thm (Szemer\'{e}di) Let $S \subset \mathbb{Z}$ of positive upper Banach density. Then $S$ contains arbitrary long arithmetic progressions. \s

\thm (Furstenberg) Let $(X,\borel,\mu, T)$ be a MPS. Let $A \in \borel$ be s.t. $\mu(A) >0$. Then for $\forall l\in \mathbb{Z}_{>0}$
\begin{align*}
\liminf_{N\rightarrow \infty} \frac{1}{N} \sum_{n=1}^N \mu(A\cap T^{-n}(A) \cap \cdots \cap T^{-(l-1)n}(A))>0
\end{align*}
\s
\s

Let $X =\{0,1\}^{\mathbb{Z}}$, $\borel$ be the Borel $\sigma$-algebra, $T=\sigma$ be the shift map.

For a set $S \subset \mathbb{Z}_{\geq 0}$, Let $x^S \in X$ be defined by
\begin{align*}
x_n^S =\begin{cases}
1 \quad \text{if} \quad n\in S\\
0 \quad \text{if} \quad \text{o/w} 
\end{cases}
\end{align*}

Let $A \in \borel$ and $A = \{x\in X:x_0=1 \}$

Observation : $n\in S \Leftrightarrow \sigma^n x^S \in A \Leftrightarrow (\sigma^n x^S)_0=1 \Leftrightarrow x_n^S =1$.

Let $\{M_m\}$ and $\{N_m\}$ be sequences s.t.
\begin{align*}
\bar{d}(S) = \lim_{m\rightarrow \infty} \frac{1}{M_m - N_m} \big|S \cap [N_m,M_m-1] \big|
\end{align*}
Let $\mu_m = \frac{1}{M_m - N_m} \sum_{n=N_m}^{M_m-1} \delta_{\sigma^n x^S}$, where $\delta_x$ is a measure on $X$ defined as
\begin{align*}
\delta_x(B) = \begin{cases}
1 \quad x\in B \\
0 \quad \text{o/w}
\end{cases}
\end{align*}

Let $\mu$ be the weak limit of a subsequence of $\mu_m$.
\s

\begin{itemize}
\item[ ] \textsl{(Reminder)}

\item \textbf{Weak Limits) : }(In fact, weak-* limits) Let $X$ be a compact metric space, Let $\mu_m$ be a sequence of Borel measures on $X$, and let $\mu$ be another Borel measure. Then $\mu_m$ weakly converges to $\mu$. In notation,
\begin{align*}
\lim-w_{m\rightarrow \infty} \mu_m = \mu
\end{align*}
if $\int f d\mu_m \rightarrow \int f d\mu$ $\forall f\in C(X)$

\item \thm (Banach-Alaoglu/Helly) Let $X$ be a compact metric space. Then $\mathscr{M}(X)$, the set of Borel probability measures endowed with the topology of weak convergence, is compact and metrizable.

In particular, there is a weakly convergent subsequence in any sequence of Borel probability measures.
\end{itemize}
\s

\lem Let $(X,\borel, \mu, \sigma)$ be as defined above is a measure preserving system.
\begin{itemize}
\item[ ] \textbf{proof sketch)} Let $B\in \borel$ Then:
\begin{align*}
\mu_m(B) &= \frac{1}{M_m-N_m} |\{ n \in [N_m, M_m-1] : \sigma^n x^S \in B \}  | \\
\mu_m(\sigma^{-1}B) &= \frac{1}{M_m-N_m} | \{n\in [N_m, M_m -1]:\sigma^n x^S \in \sigma^{-1} B \} | \\
&= \frac{1}{M_m-N_m} | \{n\in [N_m +1, M_m]:\sigma^n x^S \in B \} | \\
|\mu_m(B) - \mu_m(\sigma^{-1}B)| &\leq \frac{1}{M_m-N_m} \rightarrow 0 \quad \text{as } m\rightarrow \infty
\end{align*} 
It can be shown that we can pass to the limit and conclude $\mu (B) = \mu(\sigma^{-1}B)$.
\end{itemize}
\s

\bull \textbf{Remark :} If $B$ is a cylinder set, i.e. $\exists L \in \mathbb{Z}_{>0}$ and $\tilde{B} \subset \{0,1\}^{2L+1}$ s.t.
\begin{align*}
B = \{ x\in X : (x_{-L},\cdots,x_L) \in \tilde{B} \}
\end{align*}
then $B$ is both closed and open. Therefore $\chi_B$, the characteristic function of $B$, is continuous. Hence, the limit
\begin{align*}
\lim_{m\rightarrow \infty} \mu_m(B) = \mu(B)
\end{align*}
\s

\prop Let $S\subset\mathbb{Z}$, let $x^S$, $A$, $(X,\borel, \mu, \sigma)$ be as defined above. Let $l\in \mathbb{Z}_{>0}$. Suppose that $\exists n \in \mathbb{Z}_{>0}$ s.t.
\begin{align*}
\mu ( A \cap \sigma^{-n}(A) \cap \cdots \cap \sigma^{-n(l-1)}(A) ) >0
\end{align*}
Then $S$ contains an arithmetic progression of length $l$.

\begin{proof}
\pf Without loss of generality, we may assume that $\mu = \lim_{m\rightarrow \infty} \mu_m$(if this is not the case, we just replace $\mu_m$ with its converging subsequence).  Let $B = A \cap \sigma^{-n}(A) \cap \cdots \cap \sigma^{-n(l-1)}(A)$ and observe that $B$ is a cylinder set. Then $\mu(B) = \lim \mu_m(B)$ hence $\exists m$ s.t. $\mu_m (B) >0$.

By definition of $\mu_m$, $\exists k \in [N_m,M_m-1]$ such that $\sigma^k x^S \in B$. Hence
\begin{align*}
&\sigma^k x^S \in A, \,\, \sigma^kx^S \in \sigma^{-n}(A),\cdots,\sigma^k x^S \in \sigma^{-n(l-1)}(A) \\
\Rightarrow &\sigma^k x^S \in A, \,\, \sigma^{k+n}x^S \in A, \cdots,\sigma^{k+n(l-1)} x^S \in A
\end{align*}
and so $k,k+n,\cdots,k+n(l-1) \in S$ by earlier observation.

\eop
\end{proof}
\s

Note $A$ is also a cylinder set. Then $\mu(A)= \lim_{m} \mu_m (A)$ and
\begin{align*}
\mu(A) = \lim_m \mu_m(A) = \lim_m \frac{1}{M_m -N_m} \big| \{ n\in [N_m, M_m-1] : n\in S \} \big| = \bar{d}(S) > 0
\end{align*}
by assumption that $S$ is of positive upper Banach density, and therefore we can prove Szemer\'{e}di when assuming Furstenberg.
\s

========================================================================================================
(9th October, Tuesday)
\s

\section*{3. Poincar\'{e} recurrence, Ergodicity}

\lem Let $(X, \borel, \mu, T)$ be MPS. Let $A\in \borel$ with $\mu(A) >0$. Then $\exists n \in \mathbb{Z}_{>0}$ s.t. $\mu(A \cap T^{-n}A) >0$.
\begin{proof}
\pf Suppose $\mu(A \cap T^{-n}A) =0$ for all $n>0$. Then
\begin{align*}
\mu(T^{-k}A\cap T^{-n}A) = \mu(A \cap T^{-(n-l)}A ) =0
\end{align*}
for all $n>k\geq 0$. Hence the sets $A, T^{-1}A,\cdots$ are "almost pairwise disjoint". Then
\begin{align*}
\mu(A \cup T^{-1}A \cup \cdots \cup T^{-n}A) &= \mu(A) + \big( \mu(T^{-1}A) - \mu(T^{-1}A \cap A) \big)\\
& + \big( \mu(T^{-2}A) - \mu(T^{-2}A \cap (A \cup T^{-1}A)) + \cdots \\
& + \big( \mu(T^{-n}A) - \mu(T^{-n}A \cap (A \cup T^{-1}A \cup \cdots \cup T^{-(n-1)}A) ) \big) \\
& = (n+1) \mu(A),
\end{align*}
a contradiction if $n+1 > \mu(A)^{-1}$.

\eop
\end{proof}
\s

\thm (Poincar\'{e} recurrence) Let $(X,\borel, \mu, T)$ be MPS. Let $A \in \borel$ with $\mu(A) >0$. Then a.e. $x\in A$ returns to $A$ infinitely often. That is,
\begin{align*}
\mu(A \backslash \setlimsup{N}{n} T^{-n} A  ) =0
\end{align*}
\s

\textbf{Remark : } $x\in T^{-n} A$ $\Leftrightarrow$ $T^{n}x \in A$. So $\cup_{n=N}^{\infty} T^{-n}A$ are the points that visit $A$ at least once after time $N$. 
\s

\begin{proof}
\pf Let $A_0$ be the set of point in $A$ that never returns to $A$. We first show $\mu(A_0)=0$. Note that $\mu(A_0 \cap T^{-n}A_0) \leq \mu(A_0 \cap T^{-n}A) = \mu(\phi) =0$ for all $n>0$. By the previous lemma, we have $\mu(A_0) =0$. Note that if $x\in A \backslash \big( \setlimsup{N}{n} T^{-n} A \big)$, then there is a maximal $m\in \mathbb{Z}_{\geq 0}$ such that $T^{m}x \in A_0$. This means that
\begin{align*}
A \backslash \cap \cup T^{-n} A \subset \cup_{m=0}^{\infty}T^{-m}A_0
\end{align*}
and since $T^{-m} A_0$ has measure 0 for each $m\geq 0$, $A \backslash \cap \cup T^{-n} A$ also has measure 0.

\eop
\end{proof}
\s

However, if we are aim to show that any point of $X$(or almost every) visits a set $A$ with $\mu(A)$ infinitely often, we should prevent elements of $X$ being partitioned by orbits of $T^{-1}$. Assumption of ergodicity turns out to be enough for this.(In fact, we can make 'ergodic decomposition' for $T$ to to satisfy ergodicity on each partition - but not lecturing on this; bit tricky)
\s

\defi A MPS $(X,\borel, \mu, T)$ is called \textbf{ergodic} if $A=T^{-1}A$ implies $\mu(A) =0$ or $1$ for all $A \in \borel$.
\s

If the MPS is not ergodic, and $A \in \borel$ with $\mu(A) \in (0,1)$ s.t. $T^{-1}A =A$, then we can restrict the MPS to $A$. That is, we consider the MPS:

\quad $(A,\borel_A, \mu_A, T\big|_A)$ where $\borel_A = \{B\in \borel : B \subset A \}$, $\mu_A(B) = \mu(B)/\mu(A)$ for all $B\in \borel_A$.
\s

\thm The following are equivalent for a MPS $(X,\borel, \mu, T)$ :
\begin{itemize}
\item[(1)] $(X,\borel, \mu, T)$ is ergodic.
\item[(2)] $\mu(\setlimsup{N}{n} T^{-n} A ) =1$ for all $A\in \borel$ with $\mu(A) >0$.
\item[(3)] $\mu(A \bigtriangleup T^{-1} A) =0$ implies $\mu(A) =0$ or 1 for all $A\in \borel$.
\item[(4)] For all bounded measurable functions $f:X\rightarrow \reals$, $f = f \circ T$ a.e. implies $f$ is constant a.e.
\item[(5)] For all measurable functions $f:X\rightarrow \mathbb{C}$, $f = f \circ T$ a.e. implies $f$ is constant a.e.
\end{itemize}
\s

Each condition show different perspective to view ergodicity. The second item shows that for ergodic systems Poincar\'{e} recurrence holds in a stronger form: not only almost every point in $A$ but also almost every point in $X$ visits $A$ infinitely often. The last three conditions are often used in practice.
\begin{proof}
\pf \begin{itemize}
\item[(1)$\Rightarrow$(2)] Let $A\in \borel$ with $\mu(A) >0$ Let $B = \bigcap \bigcup T^{-n}A$, the set of points that visit $A$ infinitely often. By Poicar\'{e} recurrence(or P-recurrence), $\mu(B) \geq \mu(A) >0$. So if we show that $B = T^{-1}B$, then $\mu(B)=1$ follows by ergodicity.

\quad While, $x\in B$ $\Leftrightarrow$ $x$ visits $A$ i.o. $\Leftrightarrow$ $Tx$ visits A i.o. $\Leftrightarrow$ $Tx \in B$. So we proved $B = T^{-1}B$.
\item[(2)$\Rightarrow$(3)] Let $A\in \borel$ s.t. $\mu(A \bigtriangleup T^{-1} A) =0$. If $\mu(A) =0$, there is nothing to prove, so assume $\mu(A)>0$. Let $B= \cap \cup T^{-n} A$. By (2), we know that $\mu(B) = 1$. We show $\mu(B\backslash A) =0$, which completes the proof.

\quad Let $x\in B\backslash A$, then there is a first time $m>0$ s.t. $T^m x\in A$, hence $x\in T^{-m}A \backslash T^{-(m-1)}A$. This shows $B\backslash A \subset \bigcup T^{-m} A \backslash T^{-(m-1)A}$. But $T^{-m} A \backslash T^{-(m-1)A}$ has measure 0 because $\mu(T^{-m} A \backslash T^{-(m-1)}A)= \mu(T^{-1}A\backslash A) =0$.

\quad So we conclude $\mu(B\backslash A) =0$.
\item[(3)$\Rightarrow$(4)] Let $f:X\rightarrow \reals$ be a bounded measurable function s.t. $f= f\circ T$ almost everywhere. For any $t\in \reals$, define $A_t = \{x\in A : f(x) \leq t\}$. Then
\begin{align*}
\mu(A_t \bigtriangleup T^{-1}A_t) = \mu( \{x\in A : f(x) \leq t\} \bigtriangleup \{x\in A : f \circ T(x) \leq t\}) =0
\end{align*} 
By (3), we have $\mu(A_t)\in \{0,1\}$ for all $t$. Since $f$ was bounded, if $t$ is very small, then $\mu(A_t)=0$ and if $t$ is very large $\mu(A_t) =1$. But $t\mapsto \mu(A_t)$ is a monotone function, we have $\exists c \in \reals$ s.t. $\mu(A_t) =0$ for all $t<c$ and $\mu(A_t) =1$ for all $t>c$. Therefore we have $f(x)=c$.
\item[(4)$\Rightarrow$(1)] Let $A\in \borel$ with $A = T^{-1}A$. Then $\chi_A = \chi_A \circ T$ everywhere, so $\chi_A$ is constant a.e.
\end{itemize}
\end{proof}
\s

\textbf{Example :} The circle rotation $(\reals/\mathbb{Z},\borel, \mu, R_{\alpha})$ is ergodic if and only if $\alpha$ is irrational.

\begin{proof}
\pf Let $f:X\rightarrow \reals$ be measurable, and let $f(x) = \sum_{n\in \mathbb{Z}} a_n \exp(2\pi i nx)$. Then
\begin{align*}
f\circ R_{\alpha}(x) = f(x+\alpha) &= \sum_{n\in \mathbb{Z}}a_n \exp(2\pi in (x+\alpha)) \\
&= \sum_{n\in \mathbb{Z}} a_n \exp(2\pi in\alpha) \exp(2\pi i nx)
\end{align*}
so $f=f\circ R_{\alpha}$ is equivalent to having $a_n = a_n \exp(2\pi i n \alpha)$ for all $n$.
If $\alpha$ is irrational, then $\exp(2\pi in\alpha) \neq 1$ for all $n\neq 0$ so $a_n=0$ for all $n\neq 0$.

\eop
\end{proof}
\s

\newday

(11th October, Thursday)
\s

\section*{4. Ergodic theorems}

\thm (Mean ergodic theorem, von Neumann) Let $(x,\borel, \mu, T)$ be a MPS. Write
\begin{align*}
I = \{ f\in L^2(X) : f\circ T = f \,\,\, \text{a.e.} \}  \subset L^2(X)
\end{align*}
for the closed subspace of $T$-invariant functions. Write $P_T : L^2(X) \rightarrow I$ for the orthogonal projection. Then for every $f\in L^2(X)$, we have
\begin{align*}
\frac{1}{N} \sum_{n=0}^{N-1} f\circ T^n \rightarrow P_T f \quad \text{in } L^2(X)
\end{align*}
Here, $\frac{1}{N} \sum_{n=0}^{N-1} f\circ T^n$ called the \textbf{ergodic average}.
\s

There are two proofs for this theorem : one uses spectral theory and the the other does not. We would prove using the second approach, and sketch the first proof in the example sheet.

\s

\thm (Pointwise ergodic theorem, Birkhoff) Let $(X,\borel, \mu, T)$ be a MPS. Then for all $f\in L^1(X)$, $\exists f^* \in L^1(X)$ s.t. $f^* = f^* \circ T$ a.e. and 
\begin{align*}
\frac{1}{N} \sum_{n=0}^{N-1} f\circ T^n (x) \rightarrow f^*(x) \quad \text{a.e. in } X 
\end{align*}
\s

\textbf{Comments}
\begin{itemize}
\item[(1)] If $f\in L^2 \cap L^1$, then $f^* = P_T f$.
\item[(2)] There is an $L^p$ version of convergence in norm. That is, if $f\in L^p$, then
\begin{align*}
\frac{1}{N} \sum_{n=0}^{N-1} f\circ T^n \rightarrow f^* \quad \text{in } L^p \text{ norm}
\end{align*}
This will be proved in the example sheet.
\item[(3)] If $(X,\borel, \mu, T)$ is ergodic, then $f^*$(or $P_T f$) is constant a.e., because it is $T$-invariant.

\quad Note : $f^*(x)  = \int f^* d\mu$ a.e. By $L^1$ norm convergence, we also have
\begin{align*}
\lim_{N\rightarrow \infty} \int \frac{1}{N} \sum_{n=0}^{N-1} f\circ T^n d\mu \rightarrow \int f^* d\mu
\end{align*}
By a lemma that would follow,
\begin{align*}
\int f\circ T^n d\mu = \int fd\mu \quad \forall n, \text{ hence} \quad \int f d\mu = \int f^* d\mu
\end{align*}
Then
\begin{align*}
\frac{1}{N} \sum_{n=0}^{N-1}f(T^n x) \rightarrow \int f d\mu
\end{align*}
Can be interpreted as "time average(LHS) converges to spatial average(RHS)".
\end{itemize}
\s

\lem Let $T : X \rightarrow X$ be a measurable transformation and let $\mu$ be a probability measure. Then $\mu$ is $T$-invariant \emph{if and only if}
\begin{align}
\int f\circ T d \mu = \int f d\mu \quad \forall f \in L^1(X,\mu)  \label{*}
\end{align}
\begin{proof}
\pf  (\emph{(\ref{*}) $\Rightarrow$ measure preserving property}) : Let $A\in \borel$. Then
\begin{align*}
\mu(T^{-1} A) = \int \chi_{T^{-1} A} d\mu = \int \chi_A \circ T d\mu = \int \chi_A d\mu = \mu(A)
\end{align*}

(\emph{MPP $\Rightarrow$ (\ref{*})}) : Let $f\in L^1(X)$. If $f= \chi_A$ for some $A\in \borel$, then
\begin{align*}
\int f \circ T d\mu = \mu(T^{-1}A) = \mu (A)  = \int f d\mu
\end{align*}
(\ref{*}) hold for such $f$. Then (\ref{*}) also holds for simple functions by linearity of integration. In the case where $f$ is non-negative, let $f_n$ be a monotone increasing sequence of simple functions such that $\lim_n f_n = f$(e.g. $f_n = f\wedge n$),
\begin{align*}
\int f\circ T d\mu = \lim_n \int f_n \circ T d\mu = \lim_n \int f_n d\mu = \int fd\mu
\end{align*}
\quad In the general case, separate $f$ into positive and negative parts and conclude the proof.

\eop
\end{proof}
\s

\defi Let $(X,\borel, \mu, T)$ be a MPS. Then the \textbf{Koopman operator} is defined as : $U_T f = f\circ T$ acting on functions on $X$.
\s

\lem The Koopman operator is an isometry on $L^2(X)$. That is,
\begin{align*}
\langle f,g\rangle = \langle U_T f, U_T g \rangle
\end{align*}
\begin{proof}
\pf Apply the previous lemma for the function $f\circ \bar{g}$.
\begin{align*}
\mu(U_T f \cdot U_T \bar{g}) = \mu(U_T(f\bar{g})) = \mu(f\bar{g})
\end{align*}

\eop
\end{proof}
\s

\defi A MPS $(X,\borel, \mu, T)$ is called \textbf{invertible} if $\exists S:X\rightarrow X$, measure preserving, s.t.
\begin{align*}
S\circ T = T\circ S= id_X \quad \text{a.e.}
\end{align*}
If such a map exists, we denote it by $T^{-1} = S$. (such operator is unique up to a.s. equality)
\s

\lem If $(X,\borel, \mu, T)$ is invertible, then $U_T$ is unitary, and $U^*_T = U_{T^{-1}}$.

\begin{proof}
\pf Note : $U_{T^{-1}} \circ U_T = U_T \circ U_{T^{-1}} = id_{L^2(X)}$, so it is enough to show that $U^*_T = U_{T^{-1}}$. To do this, we need to show :
\begin{align*}
\langle U_{T^{-1}} f, g\rangle = \langle f, U_T g\rangle \quad \forall f,g, \in L^2
\end{align*}
and
\begin{align*}
\langle U_{T^{-1}} f, g\rangle = \int f\circ T^{-1} \cdot \bar{g} d\mu = \int \Big( f\circ T^{-1} \cdot \bar{g}  \Big) \circ T d\mu = \int f\cdot \bar{g} \circ T d\mu = \langle f, U_T g\rangle
\end{align*}

\eop
\end{proof}
\s

\quad Both von Neumann's and Birkhoff's theorems are easy for certain special kinds of functions. For instance, if $f\in I$ :
\begin{align*}
\frac{1}{N} \sum_{n=0}^{N-1} f\circ T^n =f
\end{align*}
Also, if $f= g \circ T -g$ for some $g$, then
\begin{align*}
\frac{1}{N} \sum_{n=0}^{N-1} f\circ T^n = \frac{1}{N}\big( g\circ T^N - g \big)
\end{align*}
It turns out that these two are the only functions that we have to worry about (in the case of von Neumann's theorem) - as presented in the following lemma.
\s

\lem Write $B = \{g\circ T-g:g\in L^2(x) \}$. Then $B^{\perp} =I$.
\s

\textit{Caution ! :} $B$ is not close in $L^2$. So we get $L^2 = I \oplus \bar{B}$, but not $L^2 = I \oplus B$. 

\begin{proof}
\pf Let $f\in L^2(X)$. Then
\begin{align*}
& f\in B^{\perp} \quad \Leftrightarrow \quad \langle f, g\circ T - g\rangle = 0 \quad \forall g \in L^2 \\
\Leftrightarrow \quad & \langle f, g\circ T \rangle = \langle f, g\rangle \quad \forall g \in L^2 \\
\Leftrightarrow \quad & \langle U^*_T f, g\rangle = \langle f, g\rangle \quad \forall g \in L^2 \\
\Leftrightarrow \quad & U^*_T f = f
\end{align*}
Now we only need to see that $U^*_T f=f$ $\Leftrightarrow$ $U_T f =f$ :
\begin{align*}
& U_T f = f \\
\Leftrightarrow \quad &\norms{f-U_T f}{}^2 = 0 \\
\Leftrightarrow \quad & \norms{f}{}^2 + \norms{U_T f}{}^2 - \langle f, U_T f\rangle - \langle U_T f,f\rangle =0 \\
\Leftrightarrow \quad & \norms{f}{}^2 + \norms{U^{*}_T f}{}^2 - \langle f, U^{*}_T f\rangle - \langle U^{*}_T f, f\rangle + \Big( \norms{U_T f} - \norms{U^*_T f}{}^2 \Big) =0 \\
\Leftrightarrow \quad & \norms{f-U^{*}_T f}{}^2 + \Big( \norms{U_T f}{}^2 - \norms{U^{*}_T f}{}^2 \Big) = 0
\end{align*}
Since $\norms{f-U^*_T f}{}^2 \geq 0$, $\norms{U_T f}{}^2 - \norms{U^*_T f}{}^2 \geq 0$(note that we do not know that $U^*_T$ is unitary, since we do not know if $T$ is invertible, but we know that $\norms{U^*_T}{op} \leq 1$), this statement is equivalent to having $f = U^*_T f$.
\end{proof}
\s

Now we are ready to prove the mean ergodic theorem.

\begin{proof}
\textbf{proof of MET) } Fix $\epsilon>0$. Let $f\in L^2$. By the lemma, $\exists g,e \in L^2$ s.t.
\begin{align*}
f= P_T f + (g\circ T -g) +e
\end{align*}
with $\norms{e}{} <\epsilon$ and
\begin{align*}
\frac{1}{N}\sum_{n=0}^{N-1}f\circ T^n = P_T f + \frac{1}{N}(g\circ T^N -g) + \frac{1}{N} \sum_{n=0}^{N-1} e\circ T^n
\end{align*}
This gives bound
\begin{align*}
\norms{\frac{1}{N}\sum_{n=0}^{N-1}f\circ T^n - P_T f}{} \leq \frac{2\norms{g}{}}{N} + \epsilon
\end{align*}
Taking $N\rightarrow \infty$ gives
\begin{align*}
\limsup_{N\rightarrow \infty} \norms{\frac{1}{N} \sum_{n=0}^{N-1} f\circ T^n - P_T f}\leq \epsilon
\end{align*}

\eop
\end{proof}
\s

\newday

(13th October, Saturday)
\s

Now we start proving mean ergodic theorem, starting with the following theorem.
\s

\thm \emph{(Maximal Ergodic Theorem, Wiener)} Let $(X,\borel, \mu, T)$ be a MPS. Let $f\in L^1$, $\alpha\in \reals_{>0}$. Let
\begin{align*}
E_{\alpha} = \{x\in X : \sup_N \frac{1}{N} \sum_{n=0}^{N-1} f(T^n x) > \alpha \}
\end{align*}
Then $\mu(E_{\alpha}) \leq \frac{1}{\alpha} \norms{f}{1}$.
\s

-the theorem is useful, because we can bound some set of particular irregularity depending on the parameter $\alpha$.

-usually, these kinds of maximal inequalities are prove using covering lemmas, e.g. using Vitalli covering lemma. This is also possible in this case, but the proof gets too long.
\s

The proof of the theorem depends on the following proposition
\s

\prop Let $(X,\borel, \mu, T)$ be a MPS. Let $f\in L^1$. Let
\begin{align*}
&f_0 =0, \quad f_1=f, \quad f_2=f\circ T + f, \quad \cdots \\
&f_n = f\circ T^{n-1} + \cdots + f\circ T + f = f_{n-1}\circ T + f \\
&F_N = \max_{n=0,\cdots,N} f_n
\end{align*}
Then $\int_{x:F_N(x)>0} f(x)d\mu(x) \geq 0$ for all $N$.
\begin{proof}
\pf  Suppose that $F_N(x)>0$. Then $F_N(x) =f_n(x)$ for some $n \in \{1,\cdots N\}$. Then $F_N(x) = f_{n-1}(Tx) + f(x) \leq F_N(Tx) +f(x)$, hence $f(x) \leq F_N(x) -F_N(Tx)$.
\begin{align*}
\int_{\{x: F_N(x)>0\}} f(x) d\mu \geq \int_{\{x:F_N(x)>0\}} (F_N(x)-F_N\circ T(x) ) d\mu(x)
\end{align*}
Note, if $F_N(x)\leq 0$, then we have $F_N(x) = 0$ and $F_N(x) - F_N(Tx) \leq 0$, so we have $F_N(x) - F_N \circ T(x) \geq 0$ on the domain $\{x: F_N(x) >0 \}$. 

\eop
\end{proof}
\s

We now prove maximal ergodic theorem.

\begin{proof}
\textbf{proof of Maximal E.T.)} Define
\begin{align*}
E_{\alpha,N} &= \{x\in X : \max_{m=0\cdots, N} \frac{1}{m} \sum_{n=0}^{m-1} f(T^n x) >\alpha   \} \\
&= \{x\in X: \max_{m=0,\cdots,N} \sum_{n=0}^{m-1} \big( f(T^n x) - \alpha \big) > 0 \}
\end{align*}
(with convention that the sum is just 0 in the case $m=0$) We apply the proposition for the function $f-\alpha$. Then
\begin{align*}
\int_{E_{\alpha, N}} \big( f(x) - \alpha \big) d\mu \geq 0
\end{align*}
Then
\begin{align*}
\norms{f}{1} \geq  \int_{E_{\alpha, N}} f(x) \geq \alpha \mu(E_{\alpha,N}) 
\end{align*}
Note that $E_{\alpha} = \bigcup_{M} E_{\alpha,M}$ is an increasing union and the inequality holds for any $N$, so $\norms{f}{1} \geq \alpha\mu(E_{\alpha})$.

\eop
\end{proof}
\s

Note that, in fact the proof in showing a somewhat stronger version of maximal ergodic theorem. Namely,
\s

\thm \emph{(Maximal Ergodic Theorem, version 2)} Let $(X,\borel, \mu, T)$ be a MPS. Let $f\in L^1$, $\alpha\in \reals_{>0}$. Let
\begin{align*}
E_{\alpha} = \{x\in X : \sup_N \frac{1}{N} \sum_{n=0}^{N-1} f(T^n x) > \alpha \}
\end{align*}
Then $\mu(E_{\alpha}) \leq \frac{1}{\alpha} \mu(f 1_{E_{\alpha}} )$.
\begin{proof}
\pf It follows from the fact $\int_{E_{\alpha, N}} f(x) \geq \alpha \mu(E_{\alpha,N}) $ for all $N\geq 0$.

\eop
\end{proof}
\s

\thm (Pointwise ergodic theorem) Let $(X,\borel, \mu, T)$ be a MPS. Let $f\in L^1$. Then $\exists f^* \in L^1$, $T$-invariant s.t.
\begin{align*}
\frac{1}{N} \sum_{n=0}^{N-1} f(T^n x) \rightarrow f^* (x) \quad \text{pointwise a.e.}
\end{align*}
\begin{proof}
\pf Fix $\epsilon >0$. Then $\exists f_{\epsilon} \in L^2$, $e_{\epsilon, 1} \in L^1$ s.t.
\begin{align*}
f = f_{\epsilon} + e_{\epsilon,1}\quad \text{and } \norms{e_{\epsilon,1}}{1} <\epsilon.
\end{align*}
Also $\exists g_{\epsilon} \in L^2$, $e_{\epsilon,2} \in L^2$ s.t.
\begin{align*}
f_{\epsilon} = P_T f_{\epsilon} + g_{\epsilon} \circ T - g_{\epsilon} + e_{\epsilon,2} \quad \text{and } \norms{e_{\epsilon,2}}{1} <\epsilon
\end{align*}
and $\exists h_{\epsilon} \in L^{\infty}$, $e_{\epsilon,3} \in L^1$ s.t.
\begin{align*}
g_{\epsilon} = h_{\epsilon} + e_{\epsilon,3}\quad \text{and } \norms{e_{\epsilon,3}}{1}<\epsilon
\end{align*}

\quad So $f = P_T f_{\epsilon} + h_{\epsilon}\circ T - h_{\epsilon} + e_{\epsilon}$, where $e_{\epsilon} \in L^1$ with$ \norms{e_{\epsilon}}{1} <\epsilon$.
\begin{align*}
\frac{1}{N} \sum_{n=0}^{N-1} f(T^n x) = P_T f_{\epsilon}(x) + \frac{1}{N} \big( h_{\epsilon}(T^n x) - h_{\epsilon}(x) \big) + \frac{1}{N}\sum_{n=0}^{N_1} e_{\epsilon}(T^n x)
\end{align*}
Let
\begin{align*}
E_{\epsilon,\alpha} = \{ x\in : \limsup_{N\rightarrow \infty} \Big| \frac{1}{N} \sum_{n=0}^{N-1} f(T^n x) - P_T f_{\epsilon}(x) > \alpha  \Big| \}
\end{align*} 
(Not same as $E_{\alpha, N}$ defined earlier)
Applying the Maximal ergodic theorem for the $f_n$ gives
\begin{align*}
\mu(E_{\epsilon,\alpha} ) \leq \frac{1}{\alpha} \norms{e_{\epsilon}}{1} \leq \frac{h\epsilon}{\alpha}
\end{align*}

\quad Let $F$ be the set of points $x$ s.t. $\frac{1}{N} \sum_{n=0}^{N_1} f(T^n x)$ does not converge at $x$. Then $F\subset \bigcup_{\alpha}F_{\alpha}$, where
\begin{align*}
F_{\alpha} = \{x\in X : \limsup_{N_1,N_2 \rightarrow \infty} \Big| \frac{1}{N} \sum_{n=0}^{N_1 -1} f(T^n x) - \frac{1}{N_2} \sum_{n=0}^{N_2 -1} f(T^n x) \Big| >2\alpha  \}
\end{align*}
Notice, $F_{\alpha} \subset E_{\epsilon, \alpha}$ for all $\epsilon>0$(?????), so $\mu(F_{\alpha}) \leq \mu(E_{\epsilon,\alpha}) \leq \frac{h \epsilon}{\alpha}$. Therefore $\mu(F_{\alpha}) =0$. We can take a countable sequence of $\alpha$'s(e.g. $(1/k)_{k\in \mathbb{N}}$) and conclude $\mu(F) = 0$.

\quad We proved that $\frac{1}{N} \sum_{n=0} f(T^n x) \rightarrow f^*(x)$ for some function $f^*$. By Fatou's lemma, we have $f^* \in L^1$, and it remains to prove $f^*(x) = f^*(Tx)$ a.e.

\quad For almost every $x$,
\begin{align*}
f^*(x) &= \lim_{N\rightarrow \infty} \frac{1}{N} \sum_{n=0}^{N-1}f(T^n x) \\
f^*(Tx) &= \lim_{N\rightarrow \infty} \frac{1}{N} \sum_{n=0}^{N-1} f(T^{n-1} x) \\
&= \lim_{N\rightarrow \infty} \frac{1}{N} \sum_{n=0}^{N-1} f(T^n x) \\
&= \lim_{N\rightarrow \infty} \frac{1}{N-1} \sum_{n=1}^{N-1} f(T^n x) \\
&= \lim_{N\rightarrow \infty} \frac{1}{N} \sum_{n=1}^{N-1} f(T^n x)
\end{align*}
Therefore $f^*(x) - f^*(Tx) = \lim_{N\rightarrow \infty} \frac{1}{N}f(x) = 0$ 

\eop
\end{proof}

==============================================================================

(this proof has faults. will repair, or delete,,,,,,,,)


\begin{proof}
\textbf{way more elegant proof) }
For simplicity, let $S_N(x) \frac{1}{N} \sum_{n=0}^{N-1} f\circ T^n (x)$.
\begin{itemize}
\item[(1)] First, let $f$ be a positive $L^1$ function. Then we may find a positive measurable function $\bar{f}(x)$  such that
\begin{align*}
&\bar{f}(x) = \liminf_{N\rightarrow \infty} S_N(x) \quad \text{a.e.}
\end{align*}
Note that $\bar{f}$ is $T$-invariant, since
\begin{align*}
S_N \circ T = \frac{1}{N}(f\circ T + \cdots + f\circ T^N) = \frac{N+1}{N} S_{N+1} - \frac{1}{N} f
\end{align*} 
Also, by Fatou's lemma,
\begin{align*}
\mu (| \bar{f} |) \leq \liminf_{N} \mu(| \frac{1}{N} \sum_{n=0}^{N-1} f\circ T^n |) = \norms{f}{1} <\infty  
\end{align*}
and therefore $\bar{f} \in L^1$ with $\int \bar{f} d\mu = \int fd\mu$. Now let $g(x) = f(x) - \bar{f}(x)$, then again $g\in L^1$, with $\int g d\mu = 0$ and 
\begin{align*}
\liminf_{N\rightarrow \infty} \frac{1}{N} \sum_{n=0}^{N-1} g\circ T^n (x) =0 \quad \text{a.e.}
\end{align*}
Now consider the set $F_q$, defined for $q\in \mathbb{Q}_{>0}$.
\begin{align*}
F_q = \{  x: \limsup_N  \frac{1}{N} \sum_{n=0}^{N-1} g \circ T^n (x) > q   \}
\end{align*}
Observe that $F_q$ is a $T$-invariant set, since
\begin{align*}
Tx \in F_q \quad &\Leftrightarrow \quad \limsup_{N} \frac{1}{N} \sum_{n=0}^{N-1} g\circ T^{n+1}(x) > q \\
&\Leftrightarrow \quad \limsup_{N} \Big( \frac{1}{N} \sum_{n=0}^{N} g\circ T^{n}(x) -\frac{1}{N} g \Big) > q \quad \Leftrightarrow \quad x\in F_q
\end{align*}
So we may repeat our arguments above to the restricted MPS $(F_q, \borel \big|_{F_q}, \mu\big|_{F_q}, T\big|_{F_q})$ and hence show that $\mu(g 1_{F_q}) = \mu(f 1_{F_q}) - \mu(\bar{f} 1_{F_q}) =0$. But by the maximal ergodic theorem(version 2), we have
\begin{align*}
& q \mu(F_q) \leq  \int_{F_q} g d\mu =0
\end{align*}
hence $\mu(F_q) =0$, and 
\begin{align*}
\mu(\{ x : |\limsup_N  \frac{1}{N} \sum_{n=0}^{N-1} g \circ T^n (x) |>0  \}) = \mu (\cap_{q\in \mathbb{Q}_{>0}} F_q) = 0
\end{align*}
We may conclude that
\begin{align*}
\limsup_N  \frac{1}{N} \sum_{n=0}^{N-1} g \circ T^n (x) = \liminf_N  \frac{1}{N} \sum_{n=0}^{N-1} g \circ T^n (x)  = 0 \quad \text{a.e.}
\end{align*}
and we see that
\begin{align*}
\lim_{N\rightarrow \infty }\frac{1}{N} \sum_{n=0}^{N-1} f \circ T^n (x) \rightarrow \bar{f} \quad \text{a.e.}
\end{align*}
with $\norms{\bar{f}}{1} \leq \norms{f}{1}$.

\item[(2)] For the general case, just divide $f$ into a non-negative part and a negative part, e.g. $f=f^+ - f^-$ and apply part (1) to show that
\begin{align*}
& \lim_{N\rightarrow \infty }\frac{1}{N} \sum_{n=0}^{N-1} f^+ \circ T^n (x) \rightarrow \bar{f}^+ \quad \text{a.e.} \\
& \lim_{N\rightarrow \infty }\frac{1}{N} \sum_{n=0}^{N-1} f^- \circ T^n (x) \rightarrow \bar{f}^- \quad \text{a.e.}
\end{align*}
and put $f^* = \bar{f}^+ - \bar{f}^-$, then we have the desired result.
\end{itemize} 

\eop
\end{proof}
\s

=======================================================================

(not done in the lecture. a question in example sheet.)

\thm (Pointwise ergodic theorem, $L^p$-version) Let $(X,\borel, \mu, T)$ be a MPS, that is $\sigma$-finite. Let $f\in L^p$. Then $\exists f^* \in L^p$, $T$-invariant s.t.
\begin{align*}
\frac{1}{N} \sum_{n=0}^{N-1} f(T^n x) \rightarrow f^* (x) \quad \text{pointwise a.e.}
\end{align*}
\begin{proof}
\pf First, assume that $f$ is a positive function. Let $(f_n)_n$ be a increasing sequence of $L^1$ functions s.t. $f_n \rightarrow f$ in $L^p$ and almost everywhere. Then by pointwise ergodic theorem for $L^1$ functions, we may find $(f^*_n)_n \subset L^1$ s.t.
\begin{align*}
\lim_{N\rightarrow \infty }\frac{1}{N} \sum_{n=0}^{N-1} f_n \circ T^n (x) \rightarrow f^*_n \quad \text{a.e.}
\end{align*}
Then $(f^*_n)_n$ also forms an increasing sequence, and therefore converges almost everywhere, say $f^*_n \rightarrow f^*$ a.e. Now, by Fatou's lemma, we have, for $n\geq m$,
\begin{align*}
\mu( (f^*_n - f^*_m)^{p} )^{1/p} \leq \liminf_{n} \Big( \mu( \frac{1}{N} \sum_{k=0}^{N-1} (f_n - f_m)  \circ T^k )  \Big)^{1/p} \leq \liminf_n \mu((f_n-f_m)^p)^{1/p}
\end{align*}
where the last inequality follows from Minkowski's inequality. Therefore, $(f^*_n)_n$ forms a Cauchy sequence in $L^p$, and in fact converges to $f^*$ in $L^p$. Also, again by Minkowski's inequality, we have $\norms{f^*_n}{p} \leq \norms{f_n}{p}$ and 
\begin{align*}
\norms{\frac{1}{N} \sum_{m=0}^{N-1} f_n \circ T^m}{p} \leq \norms{f_n}{p}
\end{align*}
so by dominated convergence theorem, we realize that $\frac{1}{N} \sum_{n=0}^{N-1} f_n \rightarrow f^*$ is in fact in $L^p$. Putting these results together, we conclude that
\begin{align*}
\frac{1}{N} \sum_{n=0}^{N-1} f \circ T^n \rightarrow f^* \quad \text{in } L^p \text{ and a.e.}
\end{align*}


\begin{center}
\begin{large}
\begin{tikzcd}
f_n \arrow[r, "L^p"] \arrow[d, "L^p"]
& f \arrow[d,  red] \\
f^*_n \arrow[r, "L^p"]
& f^*
\end{tikzcd}
\end{large}
\end{center}

\quad For general functions(not necessarily positive), divide it into a non-negative part and a negative part, and find a.e. and $L^p$ converging functions separately and add them.
\end{proof}

\newday

(16th October, Tuesday)
\s

\defi A number $x\in [0,1)$ is called \textbf{normal} in base $K$, if for every $b_1, b_2, \cdots, b_M \in \{0,\cdots, K\}$, we have :
\begin{align*}
\frac{1}{N} \big| \{ n\in \{0,\cdots, N-1\}: x_{n+1} = b+1, \cdots, x_{n+M}= b_M  \} \big| \rightarrow \frac{1}{K^M}
\end{align*}
where $x = 0.x_1x_2\cdots_{(K)}$ is a base $K$ expansion.
\s

\thm Almost every number (w.r.t. Lebesgue measure) is normal in any base $K\geq 2$. 
\begin{proof}
\pf Consider the MPS $(\reals/ \mathbb{Z},\borel, m, T_K)$($\borel$ the Borel $\sigma$-algebra, $m$ the Lebesgue measure) where $T_K(x) = K\cdot x$. From the example sheet, this is an ergodic MPS. Now fix $M$ and $b_1, \cdots, b_M$ as in the definition and consider the set
\begin{align*}
A = \Big[ (0.b_1 b_2 \cdots b_M)_{(K)}, (0.b_1\cdots b_M)_{(K)} + \frac{1}{K^M} \Big)
\end{align*}

\bull Note : $T^n x\in A \quad \Leftrightarrow \quad x_{n+1} = b_1, c\dots, x_{n+M} = b_M$

To see that $x$ is normal, we need
\begin{align*}
\frac{1}{N} \sum_{n=0}^{N-1} \chi_A(T^n x) \rightarrow \frac{1}{K^M}
\end{align*}
This holds by the pointwise ergodic theorem for almost every $x$. Since there are countably many choices for $K$, $M$, and $b,\cdots,b_M$, the theorem follows.

\eop
\end{proof}
\s

\section*{6. Unique ergodicity}

The problem with these ergodic theorems is that the converging point might differ depending on the selection of measure. To study in which cases this can be prevented, we study the uniqueness of measures that is preserved under a fixed map $T$.

\s
\defi A \textbf{topological dynamical system} is a tuple $(X,T)$, where $X$ is a compact metric space and $T:X\rightarrow X$ is a continuous map. We say that $(X,T)$ is \textbf{uniquely ergodic}, if there is only one \emph{$T$-invariant Borel probability measure} on $X$.
\s

\thm Let $(X,T)$ be a topological dynamical system. The followings are equivalent :
\begin{itemize}
\item[(1)] $(X,T)$ is uniquely ergodic.
\item[(2)] For every $f\in C(X)$, there is $c_f \in \mathbb{C}$ s.t. 
\begin{align*}
\frac{1}{N} \sum_{n=0}^{N-1} f(T^n x) \rightarrow c_f \quad \text{uniformly on } X
\end{align*}
\item[(3)]  There is a dense $A\subset C(X)$ and for each $f\in A$ there is $c_f \in \mathbb{C}$ s.t.
\begin{align*}
\frac{1}{N} \sum_{n=0}^{N-1} f(T^n x) \rightarrow c_f \quad \text{not necessarily uniformly } \forall x\in X
\end{align*}
\end{itemize}
\s

\thm \emph{(Riesz representation theorem)}  Let $X$ be a compact metric space. Then to each finite Borel measure on $X$, we associate bounded linear functional on $C(X)$ as follows : 
\begin{align*}
L_{\mu} f= \int f d\mu
\end{align*}
Then $\mu \mapsto L_{\mu}$ is a bijection from the space of finite Borel measures on $X$, $\mathscr{M}(X)$, to bounded linear functional on $C(X)$.
\s 

\cor Let $\mu_1 \neq \mu_2$ be two Borel measures on a compact metric space. Then $\mu_1 = \mu_2$ if and only if
\begin{align*}
\int f d\mu_1 = \int f d\mu_2 \quad \forall f \in C(X)
\end{align*}
\s

\defi Let $X$, $T$ be as above, let $\mu$ be a Borel measure. The push-forward of $\mu$ via $T$ is the measure
\begin{align*}
T_{*} \mu (A) = \mu(T^{-1}(A)) \quad \forall A\in \borel
\end{align*}

-This indeed defines a measure.
\s

\lem Let $X$, $T$, $\mu$ be as above. Then
\begin{align*}
\int f dT_{*}\mu = \int f \circ T d\mu
\end{align*}
for every bounded measurable function $f$.
\begin{proof}
\pf First prove this for characteristic functions of sets. Let $A\in \borel$.
\begin{align*}
\int \chi_A dT_* \mu = T_* \mu(A) = \mu(T^{-1} A) = \int \chi_{T^{-1} A } d\mu = \int \chi_A \circ T d\mu 
\end{align*}
Now use uniform class theorem to complete the proof.

\eop
\end{proof}
\s

\bull \textbf{Remark :} $\mu$ is $T$-invariant \emph{iff} $\mu = T_* \mu$.
\s

\lem Let $X, T, \mu$ be as above. Then $\mu$ is $T$-invariant \emph{iff}
\begin{align*}
\int f d\mu = \int f\circ T d\mu \quad \forall f \in C(X) \quad \cdots \cdots (\star)
\end{align*}

(we are talking about continuous functions in place of measurable functions - so is in fact enough to work with only continuous functions.)
\begin{proof}
\pf We have already seen that $\mu$ being $T$-invariant implies ($\star$).

For the other direction, note the following : suppose that $(\star)$ holds. Then $\int f dT_* d\mu = \int f\circ T d\mu = \int fd\mu$ for all $f\in C(X)$. Now by the corollary before, we have $\mu = T_* \mu$.   

\eop
\end{proof}
\s

\thm Let $(X,T)$ be a topological dynamical system. Let $(\nu_j)_j$ be a sequence of Borel probability measures on $X$. Let $(N_j) \subset \mathbb{Z}_{>0}$ be sequence s.t. $N_j \rightarrow \infty$ as $j\rightarrow \infty$. Let $\mu$ be the weak limit of a subsequence of
\begin{align*}
\frac{1}{N_j} \sum_{n=0}^{N_j -1} T^n_* \nu_j 
\end{align*}
Then $\mu$ is $T$-invariant.
\begin{proof}
\pf Fix $f\in C(X)$. Wlog, assume $w-\lim \frac{1}{N_j} \sum_{n=0}^{N_j -1} T^n_* \nu_j = \mu$.
\begin{align*}
\int f\circ T d\mu &= \lim_{j\rightarrow \infty} \int f\circ T d\Big( \frac{1}{N_j} \sum_{n=0}^{N_j-1} T^n)* \nu_j \Big)  \\
& = \lim_{j\rightarrow \infty} \frac{1}{N_j} \sum_{n=0}^{N_j-1} \int f\circ T dT^n_* \nu_j \\
& = \lim_{j\rightarrow \infty}\frac{1}{N_j} \sum_{n=0}^{N_j-1} \int f\circ T^{n+1} d\nu_j \\
& = \lim_{j\rightarrow \infty}\frac{1}{N_j} \sum_{n=1}^{N_j} \int f\circ T^{n} d\nu_j
\end{align*}
Now we can expand $\int fd\mu$ similarly
\begin{align*}
\int fd\mu = \lim_{j\rightarrow \infty} \frac{1}{N_j} \sum_{n=0}^{N_j -1} \int f\circ T^n d\nu_j
\end{align*}
then 
\begin{align*}
\Big| \int fd\mu - \int f\circ T d\mu \Big| \leq \limsup_{j\rightarrow \infty} \frac{\norms{f}{\infty} + \norms{f}{\infty}}{N_j}=0
\end{align*}
\end{proof}

\newday

(18 October, Thursday)
\s

(Example sheet distributed. Example Class at 27 Oct, 10, 24 Nov. 2pm-4pm)

(Problem 9 and 10 to be submitted before Thursday 3pm)
\s

\thm Let $(X,T)$ be a topological dynamical system. The followings are equivalent :
\begin{itemize}
\item[(1)] $(X,T)$ is uniquely ergodic.
\item[(2)] $\forall f \in C(X)$, $\exists c_f \in \mathbb{C}$ s.t. $\frac{1}{N}\sum_{n=0}^{N-1} f(T^n x) \rightarrow c_f$ uniformly.
\item[(3)] $\exists A \subset C(f)$ dense s.t. $\forall f\in A$, $\exists c_f \in \mathbb{C}$ s.t.
\begin{align*}
\frac{1}{N} \sum_{n=0}^{N-1} f(T^n x) \rightarrow c_f \quad \forall x \in X
\end{align*}
but not necessarily uniformly.
\end{itemize}
\begin{proof}
\pf \begin{itemize}
\item[(1) $\Rightarrow$ (2)] Suppose that (2) fails with $c_f = \int fd\mu$, where $\mu$ is the unique invariant measure. Then $\exists \epsilon >0$, $\exists x_1, x_2, \cdots \in X$, $\exists (N_j)_{j} \subset \mathbb{Z}$ s.t. 
\begin{align*}
\Big| \frac{1}{N_j}\sum_{n=0}^{N_j -1} f(T^n x_j) - \int f d\mu \Big| > \epsilon \quad \cdots \cdots (\star)
\end{align*}
By using Bolzanno-Weierstrass theorem to restrict to a converging subsequence whenever necessary(and using diagonal argument)(noting that ), we may suppose that
\begin{align*}
\frac{1}{N_j} \sum_{n=0}^{N_j-1} f(T^n x_j) \rightarrow a
\end{align*}
for some $a\in \mathbb{C}$. Moreover, we can also assume that
\begin{align*}
\frac{1}{N_j} \sum_{n=0}^{N_j-1} T^n_* \delta_{x_j} \rightarrow \nu
\end{align*}
for some probability measure $\nu$. By the theorem from the previous lecture, $\nu$ is $T$-invariant. Also,
\begin{align*}
\int fd\nu &= \lim_{j\rightarrow \infty} \frac{1}{N_j} \sum_{n=0}^{N_j -1} \int f dT^n_* \delta_{x_j} \\
& = \lim_{j\rightarrow \infty} \frac{1}{N_j} \sum_{n=0}^{N_j-1} f(T^n x_j) =a
\end{align*}
By $(\star)$, $|a-\int fd\mu| >\epsilon$, so $\int fd\mu \neq fd\nu$, hence $\mu \neq \nu$, a contradiction.
\item[(2) $\Rightarrow$ (3)] This implication is trivial.
\item[(3) $\Rightarrow$ (1)] Let $\mu, \nu$ be $T$-invariant probability measures. We will show that $\int fd\mu = \int fd \nu$ for all $f\in A$. Since $A$ is dense, this also holds for all $f\in C(X)$. By the corollary to Riesz representation theorem, this implies $\mu = \nu$.

\quad We know
\begin{align*}
\frac{1}{N} \sum_{n=0}^{N-1} f(T^n x) \rightarrow c_f \quad \forall x\in X
\end{align*}
By dominated convergence, has
\begin{align*}
\int f d\mu = \int \frac{1}{N} \sum_{n=0}^{N-1} f(T^n x) d\mu \rightarrow c_f
\end{align*}
Thus $\int fd\mu = c_f$.

The same argument gives $\int fd\mu = c_f$.
\end{itemize}
\eop
\end{proof}
\s

\textbf{Example :} Let $\alpha \in \reals / \mathbb{Z}$ be irrational. Then the circle rotation $(\reals/\mathbb{Z},R_{\alpha})$ is uniquely ergodic. Indeed, let $\mu$ be an $R_{\alpha}$-invariant measure. Then
\begin{align*}
& \int \exp (2\pi i n x) d\mu = \int \exp(2\pi i n R_{\alpha}(x)) d\mu \\
=& \int \exp (2\pi i n (x+\alpha)) d\mu = \exp (2\pi i n \alpha) \int \exp (2\pi i n x) d\mu
\end{align*}
Since $\alpha$ is irrational, $\exp(2\pi i n \alpha) \neq 1$ if $n\neq 0$. Then
\begin{align*}
(\dagger)\cdots\cdots \begin{cases}
\int \exp(2\pi i nx)d\mu = 0 \quad \forall n \neq 0 \\
\int 1 d\mu =1
\end{cases}
\end{align*}
Let $f$ be a trigonometric polynomial, i.e. a finite lienar combination of the functions $\exp(2\pi i nx)$, where $n\in \mathbb{Z}$. Then $(\dagger)$ implies
\begin{align*}
\int f d\mu = \int f(x) dx
\end{align*}
\begin{subproof}
\emph{Fact :} Trigonometric polynomials is dense in $C(X)$ - use Stone-Weirestrass theorem.
\end{subproof}
Therefore, $\mu$ is just a Lebesgue measure.
\s

\defi A sequence $x_1,x_2, \cdots \in [0,1)$ is said to be \textbf{equidistributed} if
\begin{align*}
\frac{1}{N} \sum_{n=0}^{N-1} f(x_n) \rightarrow \int_{\reals/\mathbb{Z}} f(x) dx \quad \forall f \in C(\reals/\mathbb{Z})
\end{align*}
\s

\textbf{Remark : } Let $0\leq a< b<1$. Then $x_1,x_2,\cdots$ is equidistributed \emph{if and only if}
\begin{align*}
\frac{1}{N} \Big| \{n\in [0,N-1] : x_n \in [a,b] \} \Big| \rightarrow a-b \quad \forall 0\leq a< b<1
\end{align*}
\s

\cor $\{n\alpha + x$ mod $1 : n\geq 0\}$ is equidistributed for all $\alpha$ irrational and $x\in [0,1)$.
\s

This is the difference between pointwise ergodic theorem and unique ergodicity - unique ergodicity shows results for all $\alpha$ irrational, while pointwise erodic theorem shows for almost every points.
\begin{proof}
\pf This follows from the previous theorem and the example.
\end{proof}

\textbf{Open Problem : }Classify the Borel probability measures on $\reals/\mathbb{Z}$ that are invariant under both $T_2$ and $T_3$.

\quad If only invariant under $T_2$, there are too many of them, so it is hopeless to classify them. However, if invariant under both $T_2$ and $T_3$, it is expected that the result is a combination of Lebesgue measure and measures derived from it.
\s

\section*{7. Equidistribution of polynomials}

\defi Let $(X,\borel, \mu, T)$ be a MPS, with $X$ a compact metric space, and $T:X \rightarrow X$ is continuous. Then $x\in X$ is called \textbf{generic w.r.t. } $\mu$ if the following holds :
\begin{align*}
& \frac{1}{N}\sum_{n=0}^{N-1} T^n_* \delta_x \rightarrow \mu \quad \text{weakly} \\
\Leftrightarrow \quad & \frac{1}{N} \sum_{n=0}^{N-1} f(T^n x) \rightarrow \int f d\mu \quad \forall f \in C(X) \quad \cdots\cdots (\star)
\end{align*}
\s

\lem $\mu$-almost every $x\in X$ is $\mu$-generic.
\begin{proof}
\pf By the pointwise ergodic theorem, for $\forall f \in C(X)$, there is a set $X_f$ with $\mu(X_f) =1$ such that ($\star$) holds. Observe that every point in $\bigcap_{f\in A} X_f$ where $A\subset C(X)$ is dense and countable is $\mu$-generic.

\eop
\end{proof}
\s

\newday

(20th October, Saturday)
\s

We seek for a generalized version of the previous corollary.
\s

\thm \emph{(Furstenbeg)} Let $(X,T)$ be a uniquely ergodic topological dynamical system. Denote by $\mu$ the invariant measure. Write
\begin{align*}
S: X \times \reals/\mathbb{Z} &\rightarrow X \times \reals/\mathbb{Z} \\
(x,y) &\mapsto (Tx, y+c(x))
\end{align*}
where $c:X\rightarrow \reals/\mathbb{Z}$ is a fixed continuous function. Then $\mu * m$, where $m$ is the Lebesgue measure on $\reals/\mathbb{Z}$ is $S$-invariant. If $\mu \otimes m$(the product measure) is $S$-ergodic, then $(X\times \reals/\mathbb{Z},S)$ is uniquely ergodic.
\s

This has name of \textbf{skew-product}.
\begin{proof}
\pf Let $f \in C(X \times \reals/\mathbb{Z})$. Then
\begin{align*}
\iint f\circ S(x,y) d\mu(x)dy  &= \int_X \int_0^1 f(Tx,y+c(x)) dyd\mu(x) = \int_X \int_{-c(x)}^{1-c(x)} f(Tx,y)dyd\mu (x) \\
&= \int_{\reals/\mathbb{Z}}\int_X f(Tx,y)d\mu(x)dy = \iint f(x,y)d\mu(x)dy
\end{align*}
So $\mu \otimes m$ is indeed $S$-invariant.
\s

Now we assume that $\mu *m$ is $S$-ergodic. We show that $(X\times \reals/\mathbb{Z},S)$ is uniquely ergodic.
\s

Recall the definition :

\begin{subproof}
\defi A point $(x,y) \in X\times \reals/\mathbb{Z}$ is \textbf{generic} if
\begin{align*}
& \frac{1}{N} \sum_{n=0}^{N-1} S^n_* \delta(x,y) \xrightarrow{\text{weakly}} \mu \otimes m \\
\Leftrightarrow \quad & \frac{1}{N}\sum_{n=0}^{N-1} f(S^n(x,y)) \rightarrow \int fd\mu dm \quad \forall f\in C(X\times \reals/\mathbb{Z})
\end{align*}
\end{subproof}
\s

Let $E$ be the set of $\mu \otimes m$-generic points. We showed last time, ergodicity implies $\mu \otimes m(E) =1$.
\s

\begin{subproof}
\textbf{Claim : } If $(x,y) \in E$, then $(x,t+y) \in E$ for all $t\in \reals/\mathbb{Z}$.

\pf Observation : $S\circ U_t = U_t \circ S$, where $U_t(x,y)=(x,t+y)$. Indeed,
\begin{align*}
S\circ U_t(x,y) = S(x,t+y) = (Tx,t+y+c(x)) =U_t(Tx,y+c(x)) =  U_t \circ S(x,y)
\end{align*}
Let $f\in C(X\times \reals/\mathbb{Z})$. Write
\begin{align*}
&\frac{1}{N} \sum_{n=0}^{N-1} f(S^n(x+ty)) = \frac{1}{N}\sum_{n=0}^{N-1} f(S^n \circ U_t(x,t)) \\
= & \frac{1}{N}\sum_{n=0}^{N-1} f(U_t \circ S^n(x,t)) \xrightarrow{N\rightarrow \infty} \iint f\circ U_t dm d\mu = \iint f dmd\mu \quad \text{since } (x,y) \in E
\end{align*}
So $(x,t+y)$ is indeed generic, i.e. $(x,t+y) \in E$.
\end{subproof}
\s

This means $E = A\times \reals/ \mathbb{Z}$ for some $A\subset X$ Borel set. Note, $\mu(A) = \mu \otimes m(E) =1$. Let $\nu$ be an $S$-invariant measure on $X\times \reals/\mathbb{Z}$.

\quad We aim to prove that $\nu(E) =1$ : Write $P$ for the projection $X\times \reals/ \mathbb{Z} \rightarrow X$. We show $P_* \nu = \mu$. It is enough to show that $P_* \nu$ is $T$-invariant. Let $B\subset X$, a Borel set. Then
\begin{align*}
P_* \nu(T^{-1}(B)) &= \nu(T^{-1} (B) \times \reals/\mathbb{Z}) = \nu (S^{-1}(B\times \reals/ \mathbb{Z}))\\
&= \nu(B\times \reals/\mathbb{Z}) = P_* \nu(B)
\end{align*}
Since $(X,\borel,\mu,T)$ was assumed to be uniquely ergodic, this forces us to have $P_* \nu =\mu$. Then $P_*\nu(A) = \mu(A) =1$, and therefore $\nu(E) = \nu(A\times \reals/ \mathbb{Z})=1$.

\quad Finally, we show that $\int f d\nu = \iint f d\mu dm$ for all $f\in C(X\times \reals/ \mathbb{Z})$ and this proves $\nu = \mu \otimes m$ : If $(x,y) \in E$, then
\begin{align*}
\frac{1}{N} \sum_{n=0}^{N-1} f(S^n (x,y)) \rightarrow \iint f d\mu dm
\end{align*}
But since $\nu(E) = 1$, this holds $\nu$-a.e. By dominated convergence,
\begin{align*}
\int f d\nu = \int \frac{1}{N} \sum_{n=0}^{N-1} f(S^n(x,y)) d\nu \rightarrow \iint fd\mu dm
\end{align*}
So we have $\int fd\nu = \iint fd\mu dm$.

\eop
\end{proof}
\s

From the theorem, we can prove a generalized version of the example from the previous lecture.
\s

\cor Let $S : (\reals/ \mathbb{Z})^d \rightarrow (\reals/ \mathbb{Z})^d$, defined by
\begin{align*}
S(x_1,x_2, \cdots, x_d)  = (x_1 +\alpha,x_2+x_1, \cdots, x_d+x_{d-1})
\end{align*}
where $\alpha \in (\reals/ \mathbb{Z}) \backslash \mathbb{Q}$ is a fixed irrational number. Then $((\reals/\mathbb{Z})^d,S)$ is uniquely invariant. 
\begin{proof}
\pf Prove by induction on $d$.
\begin{itemize}
\item $d=1$ case is the circle rotation that we already discussed.
\item Suppose $d\geq 2$ and the claim holds for $d-1$. By Furstenberg's theorem, it is enough to show that $((\reals/\mathbb{Z})^d, \borel, m^d,S)$ is ergodic ($m$ is the Lebesgue measure).

\quad Let $f$ be a bounded measurable function on $(\reals/\mathbb{Z})^d$. Then
\begin{align*}
f(x) &= \sum_{n\in \mathbb{Z}^d} a_n \exp(2\pi i n \cdot x) \quad \text{a.e.}\\ 
f(S(x)) &= \sum_{n\in \mathbb{Z}^d}  a_n \exp(2\pi i( n_1(x_1 +\alpha) + n_2(x_2+x_1) + \cdots + n_d(x_d +x_{d-1})    )) \\
&= \sum_{n\in \mathbb{Z}^d} \exp(2\pi i n_1 \alpha) a_n \exp(2\pi i((n_1 + n_2)x_1 + \cdots +(n_{d-1}+n_d)x_{d-1} + n_d x_d  )  ) \\
&= \sum_{n\in \mathbb{Z}^d} \exp(2\pi i n_1 \alpha) a_n \exp(2\pi i \hat{S}(n) \cdot x)
\end{align*}
where $\hat{S}(n) = (n_1 + n_2, n_2 + n_3, \cdots, n_{d-1}+n_d, n_d)$.

\quad Suppose $f = f\circ S$ a.e. Then $a_{\hat{S}(n)} = \exp(2\pi i\alpha n_1) a_n$. Suppose that $a_m \neq 0$ for some $m\in \mathbb{Z}^d$. We aim to show $m=0$, which implies that $f$ is constant : By Parseval's formula,
\begin{align*}
\sum_{n\in \mathbb{Z}}|a_n|^2  = \norms{f}{2}^2 < \infty
\end{align*}
This means that there are at most finite $n$'s such that $|a_m| = |a_n|$. In particular, the orbit $m,\hat{S}(m),\hat{S}^2(m),\cdots$ must be periodic. Note $(\hat{S}^j(m))_{d-1} = m_{d-1} + jm_d$. Thus $m_d =0$. Similar argument gives $m_j=0$ for all $j=2,3,\cdots,d$. Hence we should have $m=(m_1,0,\cdots,0)$ and $\hat{S}^j(m) = (m)$. We now use $a_m = \exp(2\pi i \alpha m_1) a_m$. Since $a_m \neq 0$, we must have $\exp(2\pi i\alpha m_1) =1$. As $\alpha$ is irrational, this implies $m_1 =0$. 
\end{itemize}

\eop
\end{proof}
\s

\newday

(23 October, Tuesday)
\s

(Examples Class : This Saturday 2-4pm, MR12)
\s

We meet the main theorem of the chapter.
\s

\thm \emph{(Weyl)} Let $P(x) = a_d x^d + \cdots + a_1 x + a_0$ be polynomial such that $a_j$ is irrational for at least one $j\neq 0$. Then the sequence $\{ P(n) \}_{n>0}$ is equdistributed in $[0,1)$ mod $\mathbb{Z}$.
\begin{proof}
\pf First consider the case when $a_d$ is irrational. Recall that the system $(\reals/\mathbb{Z}, S)$ is uniquely ergodic, where $S(x_1, \cdots, x_d) = (x_1 + \alpha, x_2 +x_1, \cdots, x_d + x_{d-1})$ and $\alpha \in \reals/\mathbb{Z}$ irrational.

\quad Note :
\begin{align*}
S^n \begin{pmatrix}
x_1 \\
x_2 \\
\vdots \\
x_d
\end{pmatrix} = \begin{pmatrix}
x_1 + n \alpha \\
x_2 + n x_1 + \begin{pmatrix}
n\\
2
\end{pmatrix} \alpha \\
\vdots \\
x_d + n x_{d-1} +  \cdots + \begin{pmatrix}
n \\
d-1
\end{pmatrix} x_1 + \begin{pmatrix}
n \\
d
\end{pmatrix} \alpha
\end{pmatrix}
\end{align*}
This can be proved by induction on $n$.

\quad Consider the polynomials $q_j = t(t-1)\cdots (t-j+1)/j !$. The polynomials $q_0,q_1,\cdots,q_d$ form a basis in the vector space of polynomials of degree $\leq d$. In particular, there are $x_1, \cdots,x_d,\alpha \in \reals$ such that
\begin{align*}
p(t) = \alpha q_d (t) + x_1 q_{d-1}(t) + \cdots + x_d q_0(t) \quad \cdots\cdots\cdots (\star)
\end{align*}
with $\alpha = a_d \cdot d!$ is irrational. Then there are $\alpha, x_1, \cdots,x_d \in \reals/\mathbb{Z}$ such that
\begin{align*}
p(n) = \alpha\begin{pmatrix}
n \\
d
\end{pmatrix} + x_1 \begin{pmatrix}
n \\
d-1
\end{pmatrix} + \cdots + x_d \begin{pmatrix}
n \\
0
\end{pmatrix} \quad \text{mod } \mathbb{Z}
\end{align*}
for all $n\in \mathbb{Z}$. Let $f\in C(\reals/\mathbb{Z})$ and let $g(x_1, \cdots, x_d) = f(x_d) \in (\reals/\mathbb{Z})^d$. Now we have :
\begin{align*}
\frac{1}{N} \sum_{n=0}^{N-1} f(p(n)) = \frac{1}{N} \sum_{n=0}^{N-1} g(S^n(x_1, \cdots, x_d)) \quad \cdots \cdots (\dagger)
\end{align*}
where $x_1,\cdots, x_d$ are as in $(\star)$ and $\alpha$ in the definition of $S$ is also coming from $(\star)$. By unique ergodicity, $(\dagger)$ converges to
\begin{align*}
\int \cdots \int g(t_1, \cdots, t_d) dt_1 \cdots dt_d = \int f(t)dt 
\end{align*}
for all $x \in [0,1)$ uniformly. This proves equidistribution.
\s

\textit{General case :} Let $j$ be maximal such that $a_j$ is irrational. Let $q\in \mathbb{Z}_{>0}$ be such that $qa_d, \cdots,qa_{d-1},\cdots, qa_{j+1} \in \mathbb{Z}$. Fix $b \in \{ 0,1,\cdots,q-1 \}$.

\quad Note :
\begin{align*}
p(qn+b) = a_d b^d + \cdots a_{j+1} b^{j+1} + a_j(qn +b)^j + \cdots  a_1(qn+b) + a_0 \quad \text{mod } \mathbb{Z}
\end{align*}
This is a polynomial in $n$ with irrational leading coefficient. By the special case proved earlier, $\{ p(qn + b) \}$ equidistributes for each fixed $b$.

\eop
\end{proof}
\s

This theorem was proved by number theorists way before ergodic theory was founded. There are in fact many ways to prove this theorem, e.g. using Harmonic analysis. The proof using Harmonic analysis even gives the rate of convergence to the equidistributed state, while no proof with ergodic theory does. However, there are yet more sophisticated versions of this theorem, which cannot be proved (at least as long as it is known) without using ergodic theory.
\s

\section*{8. Mixing Properties}

\defi An MPS $(X,\borel, \mu,T)$ is called \textbf{mixing} if $\forall A,B\in \borel$ and $\forall \epsilon>0$, there is $N>0$ such that
\begin{align*}
\Big| \mu(A\cap T^{-n}B) - \mu(A)\mu(B) \Big| < \epsilon \quad \forall n >N
\end{align*}
\s

\defi An MPS $(X,\borel, \mu, T)$ is called a \textbf{mixing on $k$ sets} if $\forall A_0, A_1, \cdots,A_{k-1} \in \borel$ and $\forall \epsilon >0$, there is $N$ such that
\begin{align*}
\Big| \mu(A_0 \cap T^{-n_1} A \cap \cdots \cap T^{-nk-1}A_{k-1}) - \mu(A_0) -\cdots -\mu(A_{k-1}) \Big| < \epsilon
\end{align*}
for all $n_1, \cdots, n_{k-1}$ if $n_1>N$, $n_2-n_1>N$, $\cdots$, $n_{k-1} - n_{k-2} >N$.
\s

\textbf{Open Problem :} Is there an MPS that is mixing on 2 sets but not on 3 sets?
\s

\defi
\begin{itemize}
\item A subset $S \subset \mathbb{Z}_{>0}$ has \textbf{full density} if
\begin{align*}
\frac{|S\cap [1,N]|}{N} \rightarrow 1 \quad \text{as } N\rightarrow \infty
\end{align*}
\item We say that the sequence of complex numbers $(a_n)$ \textbf{converge in density} to $a\in \mathbb{C}$ if $\{n: |a_n -a| <\epsilon \}$ has full density for all $\epsilon >0$.

In notation, write $\dlim_{n\rightarrow \infty} a_n = a$.
\item We say that $(a_n)$ \textbf{Ces\`{a}ro-converges} to $a$ if
\begin{align*}
\frac{1}{N} \sum_{n=1}^N a_n \rightarrow a \quad \text{as } N\rightarrow \infty
\end{align*}
Denote $\clim_{n\rightarrow \infty} a_n= a$
\end{itemize}
\s

\defi An MPS $(X,\borel, \mu, T)$ is \textbf{weak mixing} if $\forall A,B \in \borel$, we have
\begin{align*}
\dlim_{n\rightarrow \infty} \mu(A \cap T^{-n}B) = \mu(A) \mu(B)
\end{align*}
\s

\newday

(25th October, Thursday)
\s

\lem Let $(a_n) \subset \reals$ be a \emph{bounded sequence}. Let $a\in \reals$. Then the following are equivalent.
\begin{itemize}
\item[(1)] $\dlim_{n\rightarrow \infty} a_n =a$.
\item[(2)] $\clim_{n\rightarrow \infty} |a_n -a| =0$.
\item[(3)] $\clim_{n\rightarrow \infty} |a_n -a|^2 =0$.
\item[(4)] $\clim_{n\rightarrow \infty} a_n =a$ and $\clim a_n^2 =a^2$.
\end{itemize}
\begin{proof}
\pf \begin{itemize}
\item[(1) $\Rightarrow$ (2)] Fix $\epsilon >0$. Let $M = \sup |a_n|$. By assumption we may pick $N$ that is large enough so that
\begin{align*}
\frac{1}{N} \big| \{n\in [1,N] : |a_n -a| >\epsilon \} \big| < \epsilon
\end{align*}
We estimate
\begin{align*}
\frac{1}{N} \sum_{n=1}^N |a_n -a| &\leq \frac{1}{N} \Big( \epsilon N + 2M \cdot \epsilon N \Big)\\
&= \epsilon (1+2M)
\end{align*}
Since $M$ is constant and $\epsilon$ is arbitrary, we have
\begin{align*}
\frac{1}{N} \sum_{n=1}^N |a_n -a| \rightarrow 0 \quad \text{as } N\rightarrow \infty
\end{align*}
\item[(2) $\Rightarrow$ (1)] Fix $\epsilon >0$. Then \begin{align*}
\Big| \{n \in [1,N] : |a_n -a| > \epsilon  \} \Big| \frac{1}{\epsilon}\leq \sum_{n=1}^N |a_n-a|
\end{align*}
By (2), one gets
\begin{align*}
\frac{1}{N}|\{n\in [1,N] : |a_n -a| >\epsilon \}| \rightarrow 0 \quad \text{as } N\rightarrow \infty
\end{align*}
for each $\epsilon$. This proves (1).
\item[(1) $\Leftrightarrow$ (3)] Use same arguments. 
\item[(1),(2) $\Rightarrow$ (4)] \begin{align*}
\Big| \frac{1}{N} \sum_{n=1}^N (a_n -a) \Big| \leq \frac{1}{N}\sum_{n=1}^N |a_n -a| \rightarrow 0 \quad \text{by (2)}
\end{align*}
This implies $\clim a_n =a$.

\quad Now by the definition of $\dlim$, we see that $\dlim a_n = a$ implies $\dlim a^2_n =a^2$ and this implies $\clim |a_n^2 -a^2| \rightarrow 0$. But by our previous part of the proof, this also implies $\clim a_n^2 =a^2$.
\item[(4) $\Rightarrow$ (3)] \begin{align*}
\frac{1}{N} \sum_{n=1}^N (a_n -a)^2 &= \frac{1}{N} \sum_{n=1}^N a^2_n + \frac{1}{N} \sum_{n=1}^N a^2 - 2\frac{1}{N}\sum_{n=1}^N a a_n \rightarrow a^2 + a^2 - 2a \cdot a =0 \quad \text{as } N\rightarrow \infty
\end{align*}
\end{itemize}

\eop
\end{proof}
\s

\thm Let $(X,\borel, \mu, T)$ be an MPS. The followings are equivalent :
\begin{itemize}
\item[(1)] $(X,\borel, \mu, T)$ is weak mixing.
\item[(2)] $(X\times Y,\borel\otimes \mathscr{C}, \mu\otimes \nu, T\times S)$ is ergodic for any ergodic MPS $(Y,\mathscr{C},\nu, S)$. 
\item[(3)] $(X\times X,\borel\otimes \borel, \mu\otimes \mu, T\times T)$ is ergodic.
\item[(4)] $(X\times X,\borel\otimes \borel, \mu\otimes \mu, T\times T)$ is weak mixing.
\item[(5)] $U_T$ has no non-constant eigenfunction, i.e. if $f:X\rightarrow \mathbb{C}$ is measurable and $\lambda \in \mathbb{C}$ such that $f\circ T = \lambda f$ almost everywhere, then $f$ is constant almost everywhere. 
\end{itemize}
\s

Implication from (5) to the others involves some functional analysis, which we do not assume in this course. So we will not prove the theorem in full in the lecture. But can find a guide to the proof in the example sheet.
\s

\lem Let $(X,\borel, \mu, T)$ be an MPS. Let $\mathscr{S} \subset \borel$ be a semi-algebra($\pi$-system) that generates $\borel$. Then

\begin{itemize}
\item[(i)] $(X,\borel, \mu, T)$ is weak mixing \emph{if and only if}
\begin{align*}
\dlim_{n\rightarrow \infty} \mu(T^{-n} A \cap B) = \mu(A) \mu(B) \quad \forall A,B \in \mathscr{S}
\end{align*}
\item[(ii)] $(X,\borel, \mu, T)$ is ergodic \emph{if and only if}
\begin{align*}
\clim_{n\rightarrow \infty} \mu(T^{-n}A \cap B) = \mu(A) \mu(B)
\end{align*}
\end{itemize}
\begin{proof}
\pf (In example sheet) We aim to use Dynkin's lemma. Recall :
\begin{subproof}
\textbf{Dynkin's lemma :} A $d$-system containing a $\pi$-system $\Pi$ also contains $\sigma(\Pi)$, the $\sigma$-algebra generated by $\Pi$.
\end{subproof}
The backward implications are trivial, so we just prove forward implications here.
\begin{itemize}
\item[(i)] First,
\begin{align*}
\mathscr{D} = \{A\in \borel : \dlim_{n\rightarrow \infty} \mu(T^{-n} A \cap B) = \mu(A) \mu(B) \quad \forall B\in \mathscr{S} \}
\end{align*}
Then for any $B\in \mathscr{S}$, $A\in D$ and $(A_n)_{n} \subset D$ disjoint, we have
\begin{align*}
\dlim_{n\rightarrow \infty} \mu(T^{-n} A^c \cap B) = \mu(B) - \dlim_{n\rightarrow \infty} \mu(T^{-n} A \cap B) = \mu(A^c) \mu(B)
\end{align*}
and
\begin{align*}
\dlim_{n\rightarrow \infty} \mu(T^{-n} \cup_n A_n \cap B) = \sum_n \dlim_{n\rightarrow \infty} \mu(T^{-n} A_n \cap B) = \mu(\cup_n A_n) \mu(B)
\end{align*}
and hence $A^c \in \mathscr{D}$ and $\cup_n A_n \in \mathscr{D}$. Therefore, $D$ is a $d$-system and by Dynkin's lemma, $\mathscr{D} = \sigma(\mathscr{S}) = \borel$.

\quad Next, let 
\begin{align*}
\mathscr{D}' = \{B\in \borel : \dlim_{n\rightarrow \infty} \mu(T^{-n} A \cap B) = \mu(A) \mu(B) \quad \forall A\in \borel \}
\end{align*}
then we can show accordingly that $\mathscr{D}'$ is a $d$-system, and hence is in fact equal to $\borel$.
\item[(ii)] We may use exactly the same method to show that
\begin{align*}
\clim_{n\rightarrow \infty} \mu(T^{-n}A \cap B) = \mu(A) \mu(B) \quad \forall A,B \in \borel
\end{align*}
Now suppose $A$ is a $T$-invariant set, so $T^{-n} A = A$ for all $n\geq 0$. Application of the above formula with $B=A$ gives
\begin{align*}
\lim_{N\rightarrow \infty} \frac{1}{N}\sum_{n}\mu(A \cap A) = \mu(A) = \mu(A)\mu(A)
\end{align*}
which says that $\mu(A) \in \{0,1\}$. This proves that the MPS is ergodic.
\end{itemize}
\eop
\end{proof}
\s

We prove the theorem using this lemma.

\begin{proof}
\textbf{proof of the theorem)}
\begin{itemize}
\item[(1) $\Rightarrow$ (2)] Let $\mathscr{S}$ be the set of measurable rectangles, i.e. the set of the form $B\times C$, where $B\in \borel$, $C\in \mathscr{C}$. We write for $B_1 \times C_1$, $B_2\times C_2 \in \mathscr{S}$. Then
\begin{align*}
& \Big| \frac{1}{N}\sum_{n=1}^N \Big[ \mu \otimes \nu \big((T\times S)^{-n}(B_1 \times C_1) \cap (B_2 \times C_2) \big) - \mu \otimes \nu (B_1 \times C_1)\mu \otimes \nu (B_2 \times C_2) \Big] \Big| \\
=& \Big| \frac{1}{N} \sum_{n=1}^N \Big[ \mu(T^{-n}B_1 \cap B_2) \nu(S^{-n}C_1 \cap C_2) - \mu(B_1) \mu(B_2) \nu(C_1) \nu(C_2) \Big] \Big| \\
\leq & \frac{1}{N} \Big| \sum_{n=1}^N \Big[ \mu(T^{-n}B_1 \cap B_2) \nu(S^{-n}C_1 \cap C_2) - \mu(B_1)\mu(B_2)\nu(S^{-n}C_1\cap C_2) \Big] \Big| \\
& + \frac{1}{N} \Big| \sum_{n=1}^N \Big[ \mu(B_1)\mu(B_2)\nu(S^{-n}C_1\cap C_2) - \mu(B_1) \mu(B_2) \nu(C_1) \nu(C_2) \Big] \Big| \\
\leq & \frac{1}{N} \Big| \sum_{n=1}^N \Big[ \mu(T^{-n}B_1 \cap B_2) - \mu(B_1)\mu(B_2)\Big] \Big| \quad\quad (\text{as } \nu(S^{-n}C_1\cap C_2) \leq 1 ) \\
& + \frac{1}{N} \mu(B_1) \mu(B_2) \Big| \sum_{n=1}^N \Big[ \nu(S^{-n}C_1 \cap C_2) - \nu(C_1)\nu(C_2) \Big] \Big| \rightarrow 0 \quad \text{by ergodicity}
\end{align*}
Therefore this converges to 0 as $N\rightarrow 0$. So by the previous lemma, $(X\times Y,\borel\otimes \mathscr{C}, \mu\otimes \nu, T\times S)$ is ergodic.
\item[(2) $\Rightarrow$ (3)] (3) is a special case of (2) if we show that $(X,\borel, \mu, T)$ is ergodic. This can be seen by applying (2) with $|Y|=1$ and $S = 1_{Y}$.
\item[(3) $\Rightarrow$ (1)]  Let $A,B \in \borel$. We have
\begin{align*}
& \clim \mu \otimes \mu((T\times T)^{-n}(A\times X) \cap (B\times X)) = \mu \otimes \mu (A \times X) \cdot \mu \otimes \mu (B \times X) \\
=& \clim \mu(T^{-n}A\cap B)\mu(T^{-n}X\cap X) = \mu(A) \mu(B) \mu(X)^2 \\
=& \clim \mu(T^{-n}A\cap B) = \mu(A) \mu(B)
\end{align*}
Same argument with $A \times A$ and $B\times B$ in place of $A\times X$ and $B\times X$ gives :
\begin{align*}
\clim \mu(T^{-n}A \cap B)^2 = (\mu(A) \cdot \mu(B))^2
\end{align*}
Then by last last lemma (4) $\Rightarrow$ (1), we have
\begin{align*}
\dlim \mu(T^{-n} A\cap B)= \mu(A) \mu(b)
\end{align*}
This proves (1).
\item[(1) $\Leftrightarrow$ (4)] Use same arguments.
\end{itemize}
\end{proof}
\s

\newday

(27th October, Saturday)
\s

\begin{proof}
\textbf{proof continued)} \begin{itemize}
\item[(3) $\Rightarrow$ (5)] Suppose that $f\circ T = \lambda f$ $\mu$-a.e. for $f\in L^2$ and $\lambda \in \mathbb{C}$. Consider the function $\tilde{f}(x,y) = f(x)\bar{f}(u)$. We will show that $\tilde{f}$ is $T \times T$ invariant. We have
\begin{align*}
\tilde{f}(Tx,Ty) = f(Tx) \cdot \bar{f}(Ty) = \lambda f(x) \bar{\lambda} \bar{f}(y) = |\lambda|^2 \tilde{f}(x,y)
\end{align*}
Since $U_T$ is an isometry,
\begin{align*}
\langle f, f \rangle = \langle U_T f, U_T f\rangle = \langle \lambda f , \lambda f\rangle = |\lambda|^2 \langle f,f\rangle
\end{align*}
so $|\lambda| =1$ and $\tilde{f} \circ (T \times T) = \tilde{f}$ indeed. However, if $f$ is not constant, then $\tilde{f}$ is also not constant, so $(X\times X, \cdots)$ is not ergodic. This gives a contradiction, so $f$ need be a constant
\end{itemize}
\s

\item[(5) $\Rightarrow$ (3)] This implication requires some knowledge in functional analysis. Will not be dealt here.

\eop
\end{proof}
\s

We can in fact do better then this.
\s

\thm Let $(X,\borel, \mu, T)$ be a weak mixing MPS, and let $k\in \mathbb{Z}_{>0}$. Let $f_1,\cdots,f_k \in L^{\infty}$. Then :
\begin{align*}
\frac{1}{N} \sum_{n=1}^N U^n_T f_1 \cdots U^{2n}_T f_2 \cdots U^{kn}_T f_k \xrightarrow{L^2} \int f_1 d\mu \cdots \int f_{k} d\mu
\end{align*}
\s

\cor  $(X,\borel, \mu, T)$ be a weak mixing MPS, and let $k\in \mathbb{Z}_{>0}$. Let $f_0 \in L^{\infty}$. Then :
\begin{align*}
\frac{1}{N} \sum_{n=1}^N \int f_0 U^n_T f_1 \cdots U^{kn}_T f_k d\mu \rightarrow \int f_0 d\mu \int f_1 d\mu \cdots \int f_k d\mu
\end{align*}
In particular, let $f_0 = f_1 = \cdots = f_k = \chi_A$ for some $A\in \borel$. Then
\begin{align*}
\frac{1}{N}\sum_{n=1}^N \mu(A \cap T^{-n}A \cap \cdots \cap T^{-nk} A) \rightarrow \mu(A)^{k+1}
\end{align*}
\begin{proof}
\pf Let $g_N = \frac{1}{N} \sum_{n=1}^N U^n_T f_1 \cdots U^{2n}_T f_2 \cdots U^{kn}_T f_k$. Then the above theorem says that $g_n \rightarrow \gamma$ in $L^2$, where $\gamma$ is as in the theorem. Then it also follows that $\langle f_0, \bar{g}_n \rangle \rightarrow \langle f_0 , \bar{\gamma}\rangle$, which is exactly the statement of the corollary.

\eop
\end{proof}
\s

We need a lemma to prove the theorem.
\s

\lem \emph{(van der Corput)} Let $u_1,u_2,\cdots$ be a bounded sequence in a Hilbert space $\mathscr{H}$. For each $h \in \mathbb{Z}_{\geq 0}$, write
\begin{align*}
s_h = \limsup_{N\rightarrow \infty} \Big| \frac{1}{N} \sum_{n=1}^N \langle u_n, u_{n+h} \rangle \Big|
\end{align*}
Suppose that $\dlim_{h\rightarrow \infty} s_h =0$. Then
\begin{align*}
\norms{\frac{1}{N} \sum_{n=1}^N u_n}{} \rightarrow 0 \quad \text{as } n\rightarrow \infty
\end{align*}
\begin{proof}
\textbf{(Idea)}
\begin{align*}
\norms{\frac{1}{N}\sum_{n=1}^N u_n}{}^2 &= \frac{1}{N^2} \sum_{n_1=1}^N \sum_{n_2=1}^N \langle u_{n_1}, u_{n_2} \rangle \\
&= \frac{1}{N^2} \Big( \sum_{n=1}^N \norms{u_n}{}^2 + \sum_{h=1}^{N-1} \sum_{n=1}^{N-k} 2 \Re (\langle u_n, u_{n+h} \rangle ) \Big)
\end{align*}
\s

\pf Fix $\epsilon >0$. By the assumption, we can find $H$ such that
\begin{align*}
\frac{1}{H} \sum_{h=0}^{H-1} s_h <\epsilon
\end{align*}
Write
\begin{align*}
& \norms{\frac{1}{N} \sum_{n=1}^N u_n - \frac{1}{NH} \sum_{n=1}^N \sum_{h=1}^H u_{n-h} }{} \\ 
\leq &\frac{1}{N} \Big[ \sum_{n=1}^H \norms{u_h}{} + \sum_{n=1}^{N+H} \norms{u_n}{} \Big] < \epsilon \quad \text{if $N$ is large enough.}
\end{align*}
and
\begin{align*}
\frac{1}{NH} \norms{\sum_{n=1}^N \sum_{h=1}^H u_{n+h}}{} \leq \frac{1}{NH} \sum_{n=1}^N \norms{\sum_{h=1}^H u_{n+h}}{}
\end{align*}
Using Cauchy-Schwarz on the expression on the right hand side of the above equation, we have
\begin{align*}
\frac{1}{N^2 H^2} \norms{\sum_{n=1}^N \sum_{h=1}^H u_{n+h}}{}^2 &\leq \frac{1}{N H^2} \sum_{n=1}^N \norms{\sum_{h=1}^H u_{n+h}}{}^2 \\
& = \frac{1}{NH^2} \sum_{n=1}^N \sum_{h_1=1}^H \sum_{h_2 =1}^H \langle u_{n+h_1}, u_{n+h_2} \rangle \\
& \leq \frac{1}{H^2} \sum_{h_1=1}^H \sum_{h_2 =1}^H \frac{1}{N} \Big| \sum_{n=1}^N \langle u_{n+h_1} , u_{n+h_2} \rangle  \Big|
\end{align*}
Take $N\rightarrow \infty$, then we get
\begin{align*}
\limsup_{N \rightarrow \infty} \frac{1}{N^2 H^2} \norms{\sum_{n=1}^N \sum_{h=1}^H u_{n+h}}{}^2 &\leq \frac{1}{H^2} \sum_{h_1 =1}^H \sum_{h_2=1}^H s_{|h_1 - h_2|} \\
&\leq \frac{1}{H^2} \sum_{h=0}^{H-1} 2H \cdot s_h < 2\epsilon
\end{align*}
Combining what we have,
\begin{align*}
\lim_{N\rightarrow \infty} \norms{\frac{1}{N} \sum_{n=1}^N u_n}{} \leq \epsilon + \sqrt{2\epsilon}
\end{align*}
$\epsilon$ was chosen arbitrary, so we have the desired result.

\eop
\end{proof}
\s

\lem Let $(X,\borel, \mu, T)$ be a weak mixing measure preserving system. Let $f,g\in L^2$. Then
\begin{align*}
\dlim \langle U^n_T f,g\rangle = \int f d\mu \cdot \int \hat{g} d\mu 
\end{align*}
\begin{proof}
If we plug in characteristic functions in $f$ and $g$, this is precisely the definition of weak mixing. So we can think of this as an extension of weak mixing definition.
\end{proof}

\lem If $(X,\borel, \mu, T)$ is weak mixing, then so is $(X,\borel, \mu, T^k)$.
\begin{proof}
\pf Proof of two lemmas are exercises.
\end{proof}
\s

\end{document}
