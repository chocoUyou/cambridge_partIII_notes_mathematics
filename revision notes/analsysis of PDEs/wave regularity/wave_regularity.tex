\documentclass[12pt,a4paper]{report}


\usepackage{amsmath}
\usepackage[utf8]{inputenc}
\usepackage{amsmath}
\usepackage{amsfonts}
\usepackage{amssymb}
\usepackage{calrsfs}
\usepackage[left=2cm,right=2cm,top=2cm,bottom=2cm]{geometry}
\usepackage[mathscr]{euscript}

%%%%%%%%%%%attach pdf%%%%%%%%%%%%
\usepackage[final]{pdfpages}
%%%%%%%%%%%%%%%%%%%%%%%%%%%%%%%%%

%%%%%For writing large opertors%%%%%%%%%%%
%\usepackage{stmaryrd}
%%%%%%%%%%%%%%%%%%%%%%%%%%%%%%%%%%%%%%%%%%

%%%%%%%%%%for writing large parallel%%%%%%
\usepackage{mathtools}
\DeclarePairedDelimiter\bignorm{\lVert}{\rVert}
%%%%%%%%%%%%%%%%%%%%%%%%%%%%%%%%%%%%%%%%%%

%%%for drawing commutative diagrams.%%%%%%
\usepackage{tikz-cd}  
%%%%%%%%%%%%%%%%%%%%%%%%%%%%%%%%%%%%%%%%%%

%%%%%%%%%%for changing margin
\def\changemargin#1#2{\list{}{\rightmargin#2\leftmargin#1}\item[]}
\let\endchangemargin=\endlist 

\newenvironment{proof}
{\begin{changemargin}{1cm}{0.5cm} 
	}%your text here
	{\end{changemargin}
}

\newenvironment{subproof}
{\begin{changemargin}{0.5cm}{0.5cm} 
	}%your text here
	{\end{changemargin}
}
%%%%%%%%%%%%%%%%%%%%%%%%%%%%%

%%%%%%%%%%%%%%double rules%%%%%%%%%%%%%%%%%%%
\usepackage{lipsum}% Just for this example

\newcommand{\doublerule}[1][.4pt]{%
  \noindent
  \makebox[0pt][l]{\rule[.7ex]{\linewidth}{#1}}%
  \rule[.3ex]{\linewidth}{#1}}
%%%%%%%%%%%%%%%%%%%%%%%%%%%%%%%%%%%%%%%%%%%%%%

\begin{document}
\newcommand{\thm}{\textbf{Theorem) }}
\newcommand{\thmnum}[1]{\textbf{Theorem #1) }}
\newcommand{\defi}{\textbf{Definition) }}
\newcommand{\definum}[1]{\textbf{Definition #1) }}
\newcommand{\lem}{\textbf{Lemma) }}
\newcommand{\lemnum}[1]{\textbf{Lemma #1) }}
\newcommand{\prop}{\textbf{Proposition) }}
\newcommand{\propnum}[1]{\textbf{Proposition #1) }}
\newcommand{\corr}{\textbf{Corollary) }}
\newcommand{\corrnum}[1]{\textbf{Corollary #1) }}
\newcommand{\pf}{\textbf{proof) }}

\newcommand{\lap}{\triangle} %%Laplacian
\newcommand{\s}{\vspace{10pt}}
\newcommand{\bull}{$\bullet$}
\newcommand{\sta}{$\star$}
\newcommand{\reals}{\mathbb{R}}

\newcommand{\eop}{\hfill  \textsl{(End of proof)} $\square$} %end of proof
\newcommand{\eos}{\hfill  \textsl{(End of statement)} $\square$} %end of proof


\newcommand{\intN}{\mathbb{Z}_N}
\newcommand{\nat}{\mathbb{N}}
\newcommand{\norms}[2]{\bignorm[\big]{#1}_{#2}}
\newcommand{\abs}[1]{\big| #1 \big|}
\newcommand{\avg}{\mathbb{E}}
\newcommand{\prob}{\mathbb{P}}
\newcommand{\borel}{\mathscr{B}}
\newcommand{\EE}{\mathscr{E}}
\newcommand{\pa}{\partial}
\newcommand{\loc}{L^1_{\text{loc}}}

\renewcommand{\vec}{\underline}
\renewcommand{\bar}{\overline}

\def\doubleunderline#1{\underline{\underline{#1}}}

\newcommand{\newday}{\doublerule[0.5pt]}
\newcommand{\digression}{**********************************************************************************************}


\setlength\parindent{0pt}

\newday
\s

(not lectured, copied the handout for improve regularity)
\s

We have established the existence of a weak solution to the IBVP:
\begin{align}
\begin{array}{ll}
u_{tt}+Lu =f & \text{in }U_T\\
u=0 & \text{on } \pa^* U_T\\
u=\psi, u_t =\psi' & \text{on }\Sigma_0
\end{array} \label{111}
\end{align}
Provided $\psi\in H_0^1 (U), \psi'\in L^2(U), f\in L^2(U_T)$ a unique soltuion $u\in H^1(U_T)$ solving the equation in a weak sense exists. As in the elliptic BVP, we would like to improve the regularity of the solution so that we have a strong solution in the sense that (\ref{111}) holds pointwise almost everywhere. To motivate our arguments, let's suppose that $u\in C^{\infty}(U_T)$ solves
\begin{align*}
\begin{cases}
\begin{array}{ll}
u_{tt}-\Delta u =f & \text{in }U_T\\
u=0& \text{on } \pa^* U_T\\
u=\psi, u_t =\psi' & \text{on }\Sigma_0
\end{array}
\end{cases}
\end{align*} 
Sincee $u\in C^{\infty}$, we can differentiate the equation w.r.t. $t$ to find $u_{ttt}-\Delta u_t = f_t$. Noting also that $u_{tt}|_{t=0} = (\Delta u + f)|_{t=0} = \Delta \psi + f|_{t=0}$. we dedue that $w= u_t$ solves the equation
\begin{align*}
\begin{cases}
\begin{array}{ll}
w_{tt} - \Delta w =f_t & \text{in }U_T\\
w=0& \text{on } \pa^* U_T\\
w=\psi', w_t =\Delta \psi + f|_{t=0} & \text{on }\Sigma_0
\end{array}
\end{cases}
\end{align*}
Multiplying the equation by $e^{-\lambda t}w_t$ and integrating, we find
\begin{align*}
\int_0^{\tau} \int_U dx (w_{tt}w_t e^{-\lambda t}- \Delta w w_t e^{-\lambda t})= \int_0^{\tau} dt\int_U f_t w_t e^{-\lambda t}
\end{align*}
and therefore
\begin{align*}
& \int_0^{\tau}dt \int_U dx \Big[  \frac{d}{dt}\Big[ \Big( \frac{1}{2}w_t^2 + \frac{1}{2}|Dw|^2 e^{-\lambda t} \Big)  \Big] + \frac{\lambda}{2}(w_t^2 + |Dw|^2)\Big]\\
\leq & \frac{1}{2}\int_0^{\tau}dt \int_U (f_t^2+w_t^2)e^{-\lambda t}
\end{align*}
so
\begin{align*}
\sup_{\tau\in [0,T]} & \int_{\Sigma_{\tau}}dx (\frac{1}{2}w_{t}^2 + \frac{1}{2}|Dw|^2) + \int_0^T dt \int_U dx (\frac{1}{2} w_t^2 + |Dw|^2)\\
& \leq C_T \Big[ \int_0^T dt \int_U dx f_t^2 + \int_{\Sigma_0}dx (\frac{1}{2}w_t^2 + |Dw|^2)\Big]
\end{align*}
and
\begin{align*}
& \norms{w}{L^{\infty}((0,T); H^1(U)) } + \norms{w_t}{L^{\infty}((0,T);L^2(U))}+ \norms{w}{H^1(T)}\\
\leq & C(\norms{f}{H^1(U_T)} + \norms{\psi}{H^2(U_T)}+\norms{\psi'}{H^1(U_T)})
\end{align*}
This estimate gives us control of $u_{tt}$, $u_{tx_i}$ in $L^{\infty}((0,T); L^2(U))$. To recover the other second derivatives of $u$, the $u_{x_i x_j}$ derivatives, we note that for each time $t$,
\begin{align*}
-\Delta u = f - u_{tt}\in L^2(U),\quad u=0 \quad \text{on }\pa U
\end{align*}
so an elliptic estimate appplies and
\begin{align*}
\norms{u(t, \cdot)}{H^2(U)}\leq C (\norms{f(t, \cdot)}{L^2(U)} + \norms{u_{tt}(t, \cdot)}{L^2(U)})
\end{align*}
but we already have control on the RHS, so that
\begin{align*}
\norms{u}{L^{\infty}((0, T); H^2(U))}\leq C(\norms{f}{H^1(U_T)} + \norms{\psi}{H^2(U_T)}+ \norms{\psi'}{H^1(U_T)})
\end{align*}
Thus, by differentiating the wave equation for $u$ and performing an evergy estimat we gain $L_t^{\infty}L_x^2$-control of $u_{tt}, u_{tx_i}$. Using the wave equation as an elliptic equation for $u$ in terms of $f, u_{tt}$, we gave $L_t^{\infty}L_x^{2}$ control of $u_{x_i x_j}$.
\s

\quad To establish our improve regularity result for (\ref{111}), we essentially follow the argument above, however since we do not know our weak solution admits second derivatives, we instead have to work with the approximating seqeunce for $u$ we constructed by \emph{Galerkin's method}, and then pass to a limit.
\s

\thm \emph{(Improved regularity for the hyperbolic IBVP)} Suppose $a^{ij}, b^i,b,c\in C^2(\bar{U_T})$ and $\pa U$ is $C^2$. Then for $\psi\in H^2(U)\cap H_0^1(U)$, $\psi'\in H_0^1(U)$, $f,f_t\in L^2(U)$ the unique weak solution to (\ref{111}) in fact satisfies
\begin{align*}
& u\in H^2(U_T)\cap L^{\infty}((0,T); H^2(U))\\
& u_t \in L^{\infty}((0,T); H_0^1(U))\\
& u_{tt} \in L^{\infty}((0,T); L^2(U))
\end{align*} 
with corresponding estimates in terms of $\norms{\psi}{H^2(U)}$, $\norms{\psi'}{H^1(U)}$, $\norms{f}{f}{L^2(U_T)}$ and $\norms{f_t}{L^2(U_T)}$.
\begin{proof}
\pf We may assume $f\in C^{\infty}(\bar{U_T})$, $\psi, \psi'\in C_c^{\infty}(U)$ by approximation. We return to our Galerkin approximation. Recall
\begin{align*}
u^N(x,t) = \sum_{k=1}^N u_k^N(t)\varphi_k(x)
\end{align*} 
where we take $\varphi_k$ be the eigenfunction of the Dirichlet Laplacian and $u_k^N(t)$ solve the ODE for $k=1, 2, \cdots, N$, 
\begin{align*}
\ddot{u}_k^N (t) + \int_{\Sigma_t}\sum_{i,j}a^{ij}u_{x_i}^N (\varphi_k)_{x_j} + \sum_i b^i u_{x_i}^N \varphi_k + b \dot{u}^N \varphi_k + cu^N \varphi_k dx= (f, \varphi_k)_{L^2(U)} \cdots (\diamond)
\end{align*} 
with initial condition $u_k(0) = (\psi, \varphi_k)_{L^2(U)}$, $\dot{u}_k (0) = (\psi', \varphi_k)_{L^2(U)}$. Under our hypothesis, ($\diamond$) is a linear system of equations with coefficients in $C^2(\bar{(0,T)})$ and the RHS also is in $C^2(\bar{(0,T)})$. We can thus differentiate the system w.r.t. $t$ to find
\begin{align*}
&\frac{d^3}{dt^3}u_k^N(t) + \int_{\Sigma_t}\sum_{i,j}a^{ij}\dot{u}_{x_i}^N (\varphi_k)_{x_j} + \sum_j b^j \dot{u}_{x_i}^N \varphi_k + b \ddot{u}^N \varphi_k + c\dot{u}^N \varphi_k dx\\
=& (\dot{f}, \varphi_k)_{L^2(U)} - \int_{\Sigma_t}\sum_{i,j} \dot{a}^{ij} u_{x_i}^N (\varphi_k)_{x_j}+ \dot{b}\dot{u}^N \varphi_k + \sum_i \dot{b}^i u_{x_i}^N \varphi_k + \dot{c}u^N \varphi_k dx
\end{align*}
We multiply this expression by $\ddot{u}_k^N e^{-\lambda t}$, sum over $k=1, \cdots, N$ and integrate $\int_0^{\tau} dt$. We then deal with the LHS as for the proof of existence of weak solutions. We find
\begin{align*}
\int_0^{\tau}\int_U dx \Big(& (\frac{d^3}{dt^3}u^N)\ddot{u}^N + \sum a^{ij}\dot{u}_{x_i}^N \ddot{u}_{x_j}^N e^{-\lambda t} + \sum b^i \dot{u}_{x_i}^N \ddot{u}^N e^{-\lambda t} \\
& b (\ddot{u}^N)^2 e^{-\lambda t} + c\dot{u}^N \ddot{u}^N e^{-\lambda t} + \sum \dot{a}^{ij}u^N_{x_i}\ddot{u}^N_{x_j} e^{-\lambda t} \Big) \\
=\int_0^{\tau} \int_U dx \Big( &\dot{f}\ddot{u}^N e^{-\lambda t} - \dot{b}\dot{u}^N \ddot{u}^N e^{-\lambda t} - \sum_i \dot{b}^i u_{x_i}^N \ddot{u}^N e^{-\lambda t} \\
& - \dot{c}u^N \ddot{u}^N e^{-\lambda t}\Big)
\end{align*}
We reaarange this to obtain
\begin{align*}
\textbf{(A)} = \int_0^{\tau}dt \int_U dx \Big(& \frac{d}{dt}\Big[ e^{-\lambda t}\Big( \frac{1}{2}(\ddot{u}^N)^2 + \frac{1}{2} \sum a^{ij} \dot{u}_{x_i}^N \dot{u}_{x_j}^N + \sum \dot{u}^{ij}u^N_{x_i}\dot{u}^N_{x_j} \Big) \Big] \\
& + \lambda e^{-\lambda t}\Big( \frac{1}{2}(\ddot{u}^N)^2 + \frac{1}{2}\sum a^{ij}\dot{u}^N_{x_i} \dot{u}^N_{x_j} + \sum \dot{a}^{ij}u^N_{x_i}\dot{u}^N_{x_j} \Big)  \Big) \\
= \int_0^{\tau}dt \int_U dx \Big( & \dot{f} \ddot{u}^N e^{-\lambda t} + \sum (\ddot{a}^{ij}u^N_{x_i}\dot{u}^N_{x_j} + \frac{3}{2}\dot{a}^{ij} \dot{u}^N_{x_i} \dot{u}^N_{x_j})e^{-\lambda t} \\
& -\sum (b^i \dot{u}^N_{x_i}\ddot{u}^N + \dot{b}^i u^N_{x_i}\ddot{u}^N)e^{-\lambda t} - b(\ddot{u}^N)^2 e^{-\lambda t} \\
& - \dot{b} \dot{u}^N \ddot{u}^N e^{-\lambda t} - c\dot{u}\ddot{u}^N e^{-\lambda t} - \dot{c}u^N \ddot{u}^N e^{-\lambda t} \Big) = \textbf{(B)}
\end{align*}
We deal just with the expression \textbf{(A)}, where we can perform the $t$ integration in the first term to find
\begin{align*}
\textbf{(A)}= & e^{-\lambda \tau} \int_{\Sigma_{\tau}} dx\Big( \frac{1}{2}(\ddot(u)^N)^2 + \frac{1}{2}\sum a^{ij}\dot{u}^N_{x_i} \dot{u}^N_{x_j} + \sum \dot{a}^{ij} u^N_{x_i} \dot{u}^N_{x_j}  \Big) \\
& - \int_{\Sigma_0} dx \Big( \frac{1}{2} (\ddot{u}^N)^2 + \frac{1}{2} \sum a^{ij} \dot{u}^N_{x_i} \dot{u}^N_{x_j} + \sum \dot{a}^{ij} u^N_{x_i} \dot{u}^N_{x_j} \Big)\\
& + \int_0^{\tau} dt \int_U dx \lambda e^{-\lambda t} \Big( \frac{1}{2}(\ddot{u}^N)^2 + \frac{1}{2}\sum a^{ij}\dot{u}^N_{x_i} \dot{u}^N_{x_j} + \sum \dot{a}^{ij} u_{x_i}^N \dot{u}_{x_j}^N \Big)
\end{align*}
Now recall that by the uniform ellipticity
\begin{align*}
\sum_{i,j}a^{ij}\dot{u}^N_{x_i} \dot{u}^N_{x_j} \geq \theta |D\dot{u}|^2
\end{align*}
Moreover, by Young's inequality we can estimate
\begin{align*}
|\sum \dot{a}^{ij}u^N_{x_i}\dot{u}^N_{x_j}|\leq \frac{\theta}{4} |D\dot{u}^N|^2 + C_{\theta, a}|Du^N|^2
\end{align*}
We deduce
\begin{align*}
A \geq & e^{-\lambda \tau} \Big(\frac{1}{2}\norms{\ddot{u}^N(\tau)}{L^2(U)}^2 + \frac{\theta}{4}\norms{D\dot{u}^N(\tau)}{L^2(U)}^2 \Big) \\
& + \lambda e^{-\lambda \tau}\Big(  \frac{1}{2}\norms{\ddot{u}^N}{L^2(U_T)}^2 + \frac{\theta}{4}\norms{D\dot{u}^N}{L^2(U_{\tau})}^2 \Big) \\
& - C_1 (\norms{Du^N((\tau)}{L^2(U)}^2 + \lambda \norms{Du^N}{L^2(U_{tau})}^2 + \norms{\ddot{u}^N(0)}{L^2(U)}^2 + \norms{D\dot{u}(0)}{L^2(U)}^2)
\end{align*}
where $C_1$ is some constant depending on $\theta, a^{ij}$, but not $\lambda$. Next we consider \textbf{(B)}. By applying Young's inequality and using $e^{-\lambda t} \leq 1$ we find
\begin{align*}
\textbf{(B)} \leq C_2 (&\norms{\dot{f}}{L^2(U_{\tau})}^2 + \norms{\ddot{u}^N}{L^2(U_{\tau})}^2 + \norms{D\dot{u}^N}{L^2(U_{\tau})}^2  \\
& + \norms{Du^N}{L^2(U_{\tau})}^2 + \norms{\dot{u}^N}{L^2(U_{\tau})}^2 + \norms{u^N}{L^2(U_{\tau})}^2 )
\end{align*}
where $C_2$ is some constant depending on $a,b,c$ but not $\lambda$. Taking $\lambda$ sufficiently large we absorb the $\norms{\ddot{u}^N}{L^2(U_{\tau})}^2$ and $\norms{D\dot{u}^N}{L^2(U_{\tau})}^2$ terms on the LHS when we set \textbf{(A)=(B)}. We recall from the proof of existence of weak solutions that
\begin{align*}
\norms{Du^N}{L^2(U_T)}^2 + \norms{\dot{u}^N}{L^2U_T)}^2 + \norms{u^N}{L^2(U_T)}^2 \leq C (\norms{f}{L^2(U_T)}^2 + \norms{\psi}{H^1(U)}^2 + \norms{\psi'}{L^2(U)}^2)
\end{align*}
We also note that
\begin{align*}
\norms{D\dot{u}^N(0)}{L^2(U)}^2 \leq C \norms{D\psi'}{L^2(U)}^2
\end{align*}
Using Bessel's inequality, and using the equation ($\diamond$), we can estimate
\begin{align*}
\norms{\ddot{u}^N(0)}{L^2(U)}\leq C(\norms{\psi}{H^2(U)}^2 + \norms{\psi}{H^1(U)}^2 + \norms{f(0)}{L^2(U)}^2)
\end{align*}
Putting these all together, we find
\begin{align*}
\sup_{\tau \in [0,T]} \Big( \norms{\ddot(u)^N(\tau)}{L^2(U)}^2 + \norms{D\dot{u}^N (\tau)}{L^2(U)}^2 \Big) \leq C (\norms{\psi}{H^2(U)}^2 + \norms{\psi'}{H^1(U)}^2 + \norms{f}{L^2(U_T)}^2 + \norms{\dot{f}}{L^2(U_T)}^2)
\end{align*}
Passing to a subseqeunce, to the unique weak solution, we conclude
\begin{align*}
\ddot{u}\in L^{\infty}((0,T);L^2(U)), \quad \dot{u}\in L^{\infty} ((0, T); H_0^1(U))
\end{align*}
We can finally recover spatial regularity by noting that for almost every time $\tau$, $u(\tau)$ is a weak solution of
\begin{align*}
-\sum (a^{ij} u_{x_i})_{x_j} = -f - \ddot{u} - \sum b^i u_{x_i} - b\dot{u} - cu = \tilde{f} \quad \text{in } U
\end{align*}
with $u=0$ on $\pa U$. We have already established $\tilde{f}\in L^2(U)$, so $u(\tau)\in H^2(U)$ with
\begin{align*}
\norms{u(\tau)}{H^2(U)}^2 \leq C \norms{\tilde{f}}{L^2(U)}^2
\end{align*}
but
\begin{align*}
\norms{\tilde{f}}{L^2(U)}^2 \leq C(\norms{\psi}{H^2(U)}^2 + \norms{\psi'}{H^1(U)}^2 + \norms{f}{L^2(U_T)}^2 + \norms{\dot{f}}{L^2(U_T)}^2)
\end{align*}
Hence $u\in L^{\infty}((0, T); H^2(U))$.

\eop
\end{proof}
\s

We may iterate our argument to establish the following.
\s

\thm If $a^{ij}, b^i, b, c\in C^{kk+1}(\bar{U}_T)$, $\pa U$ is $C^{k+1}$ and
\begin{align*}
&\pa_t^i u|_{\Sigma_0}\in H_0^1 (U)\quad i=0, \cdots, k\\
&\pa_t^{k+1} u|_{\Sigma_0} \in L^2(U)\\
& \pa_t^i f \in L^2((0, T); H^{k-i}) \quad i=0, \cdots, k
\end{align*}
Then $u\in H^{k+1}(U_T)$ and $\pa_t^i u \in L^{\infty}((0, T); H^{k+1 -i}(U))$ for $i=0, \cdots, k+1$.

\quad Note that since $u_{tt}|_{\Sigma_0} = (f- Lu)|_{\Sigma_0}$ etc, the conditions at $\Sigma_0$ can be reduced to the requirement that $\psi \in H^{k+1}(U), \psi'\in H^k(U)$ and certain compatibility conditions hold at $\pa \Sigma_0$.

\end{document}
